 \documentclass[letterpaper, 10pt, draftclsnofoot, onecolumn]{IEEEtran}

\usepackage{graphicx}                                        
\usepackage{amssymb}                                         
\usepackage{amsmath}                                         
\usepackage{amsthm}                                          

\usepackage{alltt}                                           
\usepackage{float}
\usepackage{color}
\usepackage{url}

\usepackage{balance}
\usepackage[TABBOTCAP, tight]{subfigure}
\usepackage{enumitem}
\usepackage{pstricks, pst-node}

\usepackage{hyperref}
\usepackage{geometry}

\usepackage{listings}
\usepackage{color}

%\setlength{\parindent}{4em}
%\linespread{1.1}


\def\name{Steven Powers, Josh Matteson, Evan Tschuy}

%pull in the necessary preamble matter for pygments output
%\input{pygments.tex}

% The following metadata will show up in the PDF properties
\hypersetup{
  colorlinks = true,
  urlcolor = black,
  pdfauthor = {\name},
  pdfkeywords = {cs461 ``senior capstone'' requirements},
  pdftitle = {CS 461 Requirements Document},
  pdfsubject = {CS 461 Requirements Document},
  pdfpagemode = UseNone
}



\begin{document}


\begin{titlepage}
\centering
{\huge Many Voices Publishing Platform\par}
{\LARGE Requirements Document\par}
{\vspace{2mm}}
{\large D. Kevin McGrath \& Dr. Kirsten Winters -  CS461 Fall 2016\par}
{\large Steven Powers, Josh Matteson, Evan Tschuy\par}
{\vspace{10mm}}
{\large Abstract: .\par}
\end{titlepage}

\vspace{1pc}
\centerline{\sc \large Requirements}
\vspace{2pc}

NB: You are required to have your client physically sign the document, print it out, and submit it to me in hardcopy as well as ensuring there is an unsigned digital copy. Yes, I want both. \\

A Requirements Document outlines your project so that (1) you know what you are doing, and (2) your Client knows what they are getting. It is always a good idea to get that fixed ahead of time. \\

You will be creating a team name for this project, and including it in your requirements document. Remember, the client will be approving this document, so pick a name that is fun and makes you happy, but is also at least vaguely professional. I like to think of this as an exercise in naming a company -- in other words, if you were to productize this project and sell it, this team name would be the name of your company. \\

You will be making use of IEEE Std 830-1998 as the formatting guidelines for this document. You will also be required to make use of LaTeX for this document. See the course website for recorded lectures and learning material to help if you have not used LaTeX before. \\

While the IEEE 830 format may be new, this should significantly draw on knowledge learned in the prerequisite Software Engineering I course you've taken. If unsure, please discuss this with your TA or myself prior to submission or sending to the client. \\

This is very much the "what" document describing your project. The different requirements should be at the individual task level. You can do this as user stories (for something like a kanban board), or you can do it as something more at the functional level. Either way, make sure that you do these at a sufficient level of granularity such that you can mark them as complete as you go. \\

You will be creating a Gantt Chart for this. How you do this is up to you. 


\newpage
\centerline{\sc \large Signature Page}
\vspace{5pc}


\centering

\begin{tabular}{lllll}
Dr. Carlos Jensen, Client    & \_\_\_\_\_\_\_\_\_\_\_\_\_\_\_\_\_\_\_\_\_\_\_\_\_\_\_\_\_\_\_\_\_\_ & Date & \_\_\_\_\_\_\_\_\_\_\_\_\_\_\_\_\_\_\_\_\_ &  \\
                         &                                                                                  &      &                                            &  \\
Steven Powers, Developer & \_\_\_\_\_\_\_\_\_\_\_\_\_\_\_\_\_\_\_\_\_\_\_\_\_\_\_\_\_\_\_\_\_\_ & Date & \_\_\_\_\_\_\_\_\_\_\_\_\_\_\_\_\_\_\_\_\_ &  \\
                         &                                                                                  &      &                                            &  \\
Josh Matteson, Developer & \_\_\_\_\_\_\_\_\_\_\_\_\_\_\_\_\_\_\_\_\_\_\_\_\_\_\_\_\_\_\_\_\_\_ & Date & \_\_\_\_\_\_\_\_\_\_\_\_\_\_\_\_\_\_\_\_\_ &  \\
                         &                                                                                  &      &                                            &  \\
Evan Tschuy, Developer   & \_\_\_\_\_\_\_\_\_\_\_\_\_\_\_\_\_\_\_\_\_\_\_\_\_\_\_\_\_\_\_\_\_\_ & Date & \_\_\_\_\_\_\_\_\_\_\_\_\_\_\_\_\_\_\_\_\_ &  \\
                         &                                                                                  &      &                                            & 
\end{tabular}


\end{document}
