\documentclass[onecolumn, draftclsnofoot,10pt, compsoc]{IEEEtran}
\usepackage{graphicx}
\usepackage{setspace}
\usepackage{comment}
\usepackage{bigstrut}
\usepackage{geometry}
\usepackage{supertabular}
\usepackage{tabu}

\usepackage{hyperref}
\usepackage{url}
\hypersetup{
  colorlinks=true, linkcolor=blue, citecolor=blue, filecolor=blue, urlcolor=blue}

\geometry{textheight=9.5in, textwidth=7in}

% 1. Fill in these details
\def \CapstoneTeamName{		Remix}
\def \CapstoneTeamNumber{		61}
\def \GroupMemberOne{			Josh Matteson}
\def \GroupMemberTwo{			Steven Powers}
\def \GroupMemberThree{			Evan Tschuy}
\def \CapstoneProjectName{		Many Voices Platform}
\def \CapstoneSponsorCompany{	Oregon State University}
\def \CapstoneSponsorPerson{		Carlos Jensen}

% 2. Uncomment the appropriate line below so that the document type works
\def \DocType{
				%Abstract
		        Problem Statement
				%Requirements Document
				%Technology Review
				%Design Document
				%Progress Report
				}

\newcommand{\NameSigPair}[1]{\par
\makebox[2.75in][r]{#1} \hfil 	\makebox[3.25in]{\makebox[2.25in]{\hrulefill} \hfill		\makebox[.75in]{\hrulefill}}
\par\vspace{-12pt} \textit{\tiny\noindent
\makebox[2.75in]{} \hfil		\makebox[3.25in]{\makebox[2.25in][r]{Signature} \hfill	\makebox[.75in][r]{Date}}}}
% 3. If the document is not to be signed, uncomment the RENEWcommand below
\renewcommand{\NameSigPair}[1]{#1}

%%%%%%%%%%%%%%%%%%%%%%%%%%%%%%%%%%%%%%%
\begin{document}
\begin{titlepage}
    \pagenumbering{gobble}
    \begin{singlespace}
    	\includegraphics[height=4cm]{coe_v_spot1}
        \hfill
        % 4. If you have a logo, use this includegraphics command to put it on the coversheet.
        %\includegraphics[height=4cm]{CompanyLogo}
        \par\vspace{.2in}
        \centering
        \scshape{
            \huge CS Capstone \DocType \par
            {\large\today}\par
            \vspace{.5in}
            \textbf{\Huge\CapstoneProjectName}\par
            \vfill
            {\large Prepared for}\par
            \Huge \CapstoneSponsorCompany\par
            \vspace{5pt}
            {\Large\NameSigPair{\CapstoneSponsorPerson}\par}
            {\large Prepared by }\par
            Group\CapstoneTeamNumber\par
            % 5. comment out the line below this one if you do not wish to name your team
            \CapstoneTeamName\par
            \vspace{5pt}
            {\Large
                \NameSigPair{\GroupMemberOne}\par
                \NameSigPair{\GroupMemberTwo}\par
                \NameSigPair{\GroupMemberThree}\par
            }
            \vspace{20pt}
        }
        \begin{abstract}
        % 6. Fill in your abstract
		\noindent The purpose of the Many Voices Publishing Platform 
		project is to remedy problems associated with the current textbook market 
		and standard expectations that come with textbooks. 
		The Many Voices Publishing Platform was developed to alleviate 
		costs of textbooks for students and provide instructors with the ability 
		to create their own. Instructors can use the open platform for 
		collaboration between content creators, and other instructors in order 
		to choose the specific focus of their course materials. 
		This platform was created using a custom Git back-end integration and Aurelia 
		as a front-end JavaScript framework to provide the ability to implement a 
		collaborative platform for authoring.
        \end{abstract}
    \end{singlespace}
\end{titlepage}
\newpage
\pagenumbering{arabic}
\tableofcontents
% 7. uncomment this (if applicable). Consider adding a page break.
%\listoffigures
%\listoftables
\clearpage


% 8. now you write!
\section{Revision Log}
\begin{flushleft}
\tablehead{}
\begin{supertabular}{|p{3cm}|p{3cm}|p{3cm}|p{7cm}|}
\hline
Name & Change Number & Date & Description of Change  
\\\hline
Steven Powers & 1 & 2/13/2017 & 
\begin{itemize}
\item{Updated the Problem Statement to remove references to Ward Cunningham's Federated Wiki.}
\item{Updated document to new design.}
\item{Added clarifying language:}
\item{downloadable eBook (to) downloadable eBook (PDF)} 
\item{Changed reference from GitHub to Git} 
\item{Removed reference to editing like Wikipedia 
	This was intended to work with Ward Cunningham's Federated Wiki).} 
\item{their customizations back to other professors (to) their customizations to their shared materials}
\end{itemize}
\\ \hline

\end{supertabular}
\end{flushleft}



\section{Problem Definition}
\noindent Today, textbooks are filled with yearly updates that 
are often released with the minimum number of changes needed to 
justify a new edition. These textbooks repeatedly cost in the range of 
hundreds of dollars and are often filled with poor or even incorrect 
information. 
Students are often left with poor to understand information that is often more
complicated than previous versions of the textbook.
In the United States, most textbook materials are decided by a few 
states that choose what will appear in textbooks published nation wide. 
Most professors will simply choose the book provided to them for 
review that satisfies their requirements the best, without considering 
the costs that will be passed onto students. \\

\noindent Furthermore, the addition of online course materials is used 
to keep students buying new temporary subscription services that used to 
be free. 
A professor can either choose a textbook from a catalog, one provided 
for them, or spend months to years writing one of their own. 
If a professor chooses to write their own textbook, 
they will have to seek publishers, other contributors, and have 
their work peer reviewed before it can finally be released to store 
shelves. 
If a professor chooses a book through a catalog or one provided for 
them, the pre-written textbooks often contain unnecessary chapters, 
or can even be missing key sections on topics viewed as critical. 
This problem can further result in additional class materials being 
purchased by students to supplement the missing information. 
Students and professors are left with inadequate materials, damaging 
the scholastic environment. \\

\section{Proposed Solutions}


\noindent The Many Voices platform, instead, allows professors to create 
their own textbooks by using crowd-sourced content. These created textbooks
will reduce the costs of publishing, allowing students to purchase, or 
otherwise receive the textbook at a reduced cost. This monetary benefit
will allow professors to update their courses more frequently without 
passing on large costs to the students. This benefit can take the form of 
a professor that previously would not create their own course textbooks, or 
reducing the costs of available textbooks that the professor might choose.
Content creators are able to submit materials of all different formats, 
including: images, small pieces of text, pages, or even entire chapters. 
The submitted materials are able to be reviewed by editors and 
other contributors to prevent invalid materials from being used.
A knowledgeable professor can contribute a chapter or unit on 
their specialty, and others are able to modify obtained materials 
in the way they see fit. These materials may be offered for free 
or for a nominal fee, determined by the content creator. \\

\noindent Once materials are acquired, anyone is able to create a more 
affordable textbook available in either a downloadable eBook (PDF) or in 
printed form to create the exact textbook they want for their classes or 
for any purpose they desire. 
These materials are able to be used with version control software such 
as Git, that allow for users to fork and branch their changes to 
satisfy their goals.
Additionally, the platform features a way for editors to review 
content their content and make edits as they please. \\

\noindent The platform also features a way to search for and find 
documents relevant to the users interests. 
Through a variety of search criteria, the requested topic or subject will
turn up results prioritized by relevance and credibility. By default, the 
platform will offer the most used and/or reviewed materials
based on entered search terms. Users will be able to narrow down their 
options by specifying date ranges, certain contributors, editor review 
score, and more choices to make finding the right document an easy 
process. 
The finished platform will allow users to see existing 
textbooks, customize them using their own or third party materials, and 
contribute their customizations to their shared materials. \\

\section{Performance Metrics}

\noindent The platform will be viewed as a success if the platform 
allows professors to submit, edit, publish, and compile materials for 
the creation of textbooks and other instructional materials. 
Our goal is to ensure the platform meets our client's project 
expectations, which includes the ability to combine different materials
into a textbook, in addition to the defined needs, 
and functions as according to our development plan. \\

\noindent We will, in our best efforts, plan to meet at minimum every two weeks 
with our client to discuss current progress of project goals and 
proposed functionality for the application. 
Finally, before beginning work on major project milestones and certain 
aspects of the platform, we will discuss the overall status of the 
project. We will further inquire as to whether our client has any additional 
insight to assist in meeting project expectations as desired. 
After reaching a reasonable agreement, we will continue to adjust and 
shape the platform where necessary. 
We will further ensure that the development taking place on the platform 
is to the best of our abilities.\\

\noindent A successful implementation of the platform will include the ability 
to combine materials of various types into a full document for use as a textbook.

\end{document}
