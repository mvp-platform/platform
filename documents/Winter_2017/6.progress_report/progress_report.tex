\documentclass[onecolumn, draftclsnofoot,10pt, compsoc]{IEEEtran}
\usepackage{graphicx}
\usepackage{setspace}
\usepackage{comment}
\usepackage{bigstrut}
\usepackage{geometry}
\usepackage{supertabular}
\usepackage{tabu}
\usepackage{hyperref}
\usepackage{url}
\hypersetup{
  colorlinks=true, linkcolor=blue, citecolor=blue, filecolor=blue, urlcolor=blue}
\geometry{textheight=9.5in, textwidth=7in}

\usepackage{color}
\definecolor{editorGray}{rgb}{0.95, 0.95, 0.95}
\definecolor{editorOcher}{rgb}{1, 0.5, 0} % #FF7F00 -> rgb(239, 169, 0)
\definecolor{editorGreen}{rgb}{0, 0.67, 0} % #007C00 -> rgb(0, 124, 0)

\usepackage{listings}
\lstdefinelanguage{JavaScript}{
  morekeywords={typeof, new, true, false, catch, function, return, null, catch, switch, 
  var, if, in, while, do, else, case, break, addChapter, await},
  morecomment=[s]{/*}{*/},
  morecomment=[l]//,
  morestring=[b]",
  morestring=[b]'
}

\lstset{%
    % Basic design
    backgroundcolor=\color{editorGray},
    basicstyle={\small\ttfamily},   
    frame=l,
    % Line numbers
    xleftmargin={0.75cm},
    numbers=left,
    stepnumber=5,
    firstnumber=1,
    numberfirstline=true,
    % Code design   
    keywordstyle=\color{blue}\bfseries,
    commentstyle=\color{red}\ttfamily,
    ndkeywordstyle=\color{editorOcher}\bfseries,
    stringstyle=\color{editorGreen},
    % Code
    language=JavaScript,
    alsodigit={.:;},
    tabsize=2,
    showtabs=false,
    showspaces=false,
    showstringspaces=true,
    extendedchars=true,
    breaklines=true
}


% 1. Fill in these details
\def \CapstoneTeamName{		Remix}
\def \CapstoneTeamNumber{		61}
\def \GroupMemberOne{			Josh Matteson}
\def \GroupMemberTwo{			Steven Powers}
\def \GroupMemberThree{			Evan Tschuy}
\def \CapstoneProjectName{		Many Voices Platform}
\def \CapstoneSponsorCompany{	Oregon State University}
\def \CapstoneSponsorPerson{		Carlos Jensen}

% 2. Uncomment the appropriate line below so that the document type works
\def \DocType{
        %Problem Statement
				%Requirements Document
				%Technology Review
				%Design Document
				Progress Report
				}

\newcommand{\NameSigPair}[1]{\par
\makebox[2.75in][r]{#1} \hfil 	\makebox[3.25in]{\makebox[2.25in]{\hrulefill} \hfill		\makebox[.75in]{\hrulefill}}
\par\vspace{-12pt} \textit{\tiny\noindent
\makebox[2.75in]{} \hfil		\makebox[3.25in]{\makebox[2.25in][r]{Signature} \hfill	\makebox[.75in][r]{Date}}}}
% 3. If the document is not to be signed, uncomment the RENEWcommand below
\renewcommand{\NameSigPair}[1]{#1}

%%%%%%%%%%%%%%%%%%%%%%%%%%%%%%%%%%%%%%%
\begin{document}
\begin{titlepage}
    \pagenumbering{gobble}
    \begin{singlespace}
    	\includegraphics[height=4cm]{coe_v_spot1}
        \hfill
        % 4. If you have a logo, use this includegraphics command to put it on the coversheet.
        %\includegraphics[height=4cm]{CompanyLogo}
        \par\vspace{.2in}
        \centering
        \scshape{
            \huge CS Capstone \DocType \par
            {\large\today}\par
            \vspace{.5in}
            \textbf{\Huge\CapstoneProjectName}\par
            \vfill
            {\large Prepared for}\par
            \Huge \CapstoneSponsorCompany\par
            \vspace{5pt}
            {\Large\NameSigPair{\CapstoneSponsorPerson}\par}
            {\large Prepared by }\par
            Group\CapstoneTeamNumber\par
            % 5. comment out the line below this one if you do not wish to name your team
            \CapstoneTeamName\par
            \vspace{5pt}
            {\Large
                \NameSigPair{\GroupMemberOne}\par
                \NameSigPair{\GroupMemberTwo}\par
                \NameSigPair{\GroupMemberThree}\par
            }
            \vspace{20pt}
        }
        \begin{abstract}
        % 6. Fill in your abstract
		\noindent This document summaries the progress that the Remix team
		has made on the Many Voices Publishing Platform for the client
		Dr. Carlos Jensen. Additionally this document provides a week by
		week summary of work performed, as overviews of progress that has been made
		throughout Winter Term.
        \end{abstract}
    \end{singlespace}
\end{titlepage}
\newpage
\pagenumbering{arabic}
\tableofcontents
% 7. uncomment this (if applicable). Consider adding a page break.
%\listoffigures
%\listoftables
\clearpage


% 8. now you write!
\section{Revision Log}

\tablehead{}
\begin{supertabular}{|p{3cm}|p{3cm}|p{3cm}|p{7cm}|}
\hline
Name & Change Number & Date & Description of Change
\\\hline
Steven Powers & 1 & 2/17/2017 & Finalized Progress Report for half way point through 
		winter term. Fixed spelling mistakes, added code listings, adjusted content, added images.
\\\hline
\end{supertabular}


\section{Project Purpose}
\noindent A modern textbook is updated frequently, perhaps even yearly,
and can cost in the range of hundreds of dollars. Students are often left
to attempt to understand poorly worded, even incorrect information from a textbook
often chosen from those sent to a professor for review by the publisher.
This can lead to better works with less aggressive sales tactics not being
made available, or even known.
Another choice would be for a professor to write their own textbook.
However, this is a process that takes months of endless research and
time spent, and on top of that will require peer review and publishing
before it can be released. \\

\noindent The Many Voices platform offers to put an end to this massive,
slow, expensive cycle.  Instead of a textbook being a single document
written by one professor, we seek to re-imagine the textbook as instead
a collection of content written by professors from around the world
that are useful for a particular class. A knowledgeable professor can
contribute a few chapters on their specialty, without needing to write
an entire textbook around it. \\

\noindent Professors wishing to use this content can then modify it for
their uses in the classroom. The material will be hosted in such a way
as to provide the ability to "fork" content, or create content based
off of it. The platform will provide a way to search for and find content,
prioritized by relevance and credibility as determined by other users;
the most popular material will be shown with the most prominence.

\newpage
\section{Winter Term Weekly Summary}
\noindent Steven:\\
\noindent Here is our previous term Positives, Deltas, and Actions chart which we felt is
still important to our progress report for Winter Term as it shows some additional history
to our project development. \\

\begin{flushleft}
\begin{tabular}{|p{0.05\linewidth}|p{0.285\linewidth}|p{0.285\linewidth}|p{0.285\linewidth}|}
\hline
	Week & Positives & Deltas & Actions \\ \hline

	3
	& Got the initial draft of the problem statement done after meeting last
	week with our client.
	& Were unable to set up a meeting with our client, and thus had no way of
	getting feedback on the draft before submission.
	& We will talk to our client via email about possible meeting times, either
	in person or remote, that may possibly work for everyone. \\ \hline

	4
	& Figured out a time for meeting with our TA.
	& Need to hear back from our client.
	& We will wait another day or two and then send a new polite email. \\ \hline

	5
	& Finished revising the problem statement, got it signed.
		Began the requirements document.
	& N/A
	& N/A\\ \hline

	6
	& Revised Requirements document via feedback from TA. Got client signature on Requirements.
		Attended writing workshop.
	& Need to be on top of things regarding contacting our client as
		he is difficult to contact.
	& We will send him another email asking for feedback on the requirements. \\ \hline

	7
	& Began technology review. Revised requirements document for possible re-grading.
	& Need better team work schedule to get times to work on documents
	together rather than separately.
	& We will discuss possible extra meeting times in addition to our weekly meeting
	with our TA where we can work for several hours uninterrupted. \\ \hline

	8
	& Finished technology review (after getting extension to Wednesday).
	& Need to be on top of documents to finish them well in advance of due dates!
	& We will personally push ourselves to get work done sooner rather than later. \\ \hline


	9 \hspace{3mm} \& 10
	& Spent Thanksgiving Week getting rested. Finished design document.
		Finished progress update document/finished presentation.
	& Need to get client signature for design document!
	& We will be emailing Carlos. With the new term we hope
		to set up a weekly meeting with him.  \\ \hline

\end{tabular}
\end{flushleft}

\newpage
\section{Steven Term Progress}

\subsection{Weekly Recapping}
\noindent For this first week we were getting back into the groove of things after Winter break.
Josh and I were able to meet with Carlos the week prior and talked about arranging meetings 
going forward each week. We weren't able to get a whole lot of work done this week, 
but we planned for meeting each week at least a few hours a couple days of the week so we can
continue making progress. We planned to meet next week and work on the framework and user stories. \\

\noindent The second week of the term we began our weekly meetings, though our client had to cancel
meeting for the first two weeks. We also setup our meetings with Jon, which was
actually pretty difficult to find a time that worked for all four of our schedules.
We plan to continue working on the User Stories and Framework so we are able
to make progress towards a working prototype. \\

\noindent The third week we were to begin our weekly meetings with our client.
Our client recently transferred positions and our weekly appointment was
unfortunately lost. This resulted in our weekly time slot being given away, so we had
o find a new time that worked for all of us. We decided for 9:00AM Mondays, which was 
unfortunate because I had planned on sleeping in these days. Our client also wanted to
instead meet biweekly rather than every week. We talked over the past terms documents and 
possible upcoming documents that we would be seeing in the class and that we should be 
focusing on development throughout this term. During this week we also worked on the backend 
system and testing of the include and input commands with our setup which is working well at 
at this time. I looked into security features of LaTeX to protect our LaTeX documents from 
succumbing to common exploits which include (escaping the document and accessing the shell).
Josh and I also worked on getting the framework up and running (TS Lint was giving us issues so
we ended up switching to the ES Next variant of JavaScript). \\

\noindent On the fourth week of the term we met with Dr. Jensen and discussed the backend 
of the system as well as some plans for how to model the front end of the system. Dr. Jensen 
suggested using low fidelity / medium fidelity to receive feedback from users instead of 
focusing on getting to high fidelity and having users review that as they will be less 
critical of the whole system and instead focus on pixel alignment. Josh and I need to get more 
work done on Aurelia and get a basic user interface up and running and try working towards 
in page rendering of the resulting PDF from the back end system. I began working on some 
prototypes using a two column view, with our PDF rendering taking place to the right of the
working area. \\


\noindent The fifth week of the term was a week we had off from meeting with Dr. Jensen, 
we used this time to work on developing some of the required features that we need. This
included Josh and I working on the PDF in browser rendering with Aurelia.
Josh finished this task up separately. I worked on Prototype development to have Dr. 
Jensen review and provide feedback on during next weeks meeting. I wasn't able to get as 
many prototypes drawn as I would have liked, but there is still a good chunk of design that
Dr. Jensen could critique. Next week I planned to show Dr. Jensen our designs and take in 
any feedback he might have. I also plan to work on revising all of our documents and get them
placed into our OneNote document that I have prepared for our group. \\

\noindent The sixth week of the term, the week that this progress report was due. We met with 
Dr. Jensen and he had lots of feedback about our designs, mainly in ways to simplify the interface 
for the users while also making it easier to develop. We talked about rendering the PDF only 
when needed (as in active editing) instead of constantly. We also talked about the possibility 
of using a tabbed system to only render when the user wants to, which will reduce our server calls. \\

\noindent I have continued working on our document revisions, as well as our Progress
Report materials (video, presentation materials, and report), all of which are currently in near
final forms as I write this. Something that will be better next week is availability, 
as this week has a lot of midterms for various classes. Josh and I plan to integrate our PDF 
viewer into our application and work on making progress on other areas of the application. \\

\subsection{What's Left on the Frontend}

Josh will go into greater detail on some of the below items in his section, 
but I will briefly list and expand upon those that he does not.

\begin{enumerate}
    \item
    \textbf{Integration of PDF Viewer and Application} \\
	As of right now we have a functioning PDF viewer built with a special branch of PDF.JS that 
	is made for Aurelia. We have implemented a separate PDF viewer page at the moment, and need
	to integrate this into our main application. \\

    \item
    \textbf{Finalizing Layout Design \& User Testing} \\
	We met with our client earlier this week with a few layout prototypes and were able to get
	good feedback on our designs, as Dr. Jensen is wanting to write a book, his feedback is valuable,
	though he said that we might come up with something better, or have other users desire something
	else. We plan to iterate on our designs post feedback and create medium fidelity prototypes with
	a tool similar to Balsamiq, a prototype editor. \\


	\item
	\textbf{Working Scrap Editor} \\
	This item is one of the most important features, as the user will be editing all of their work
	so it is important for us to provide a familiar interface to what a user might expect. Formatting
	isn't a requirement for this application, but it is something we can hopefully integrate as a 
	stretch goal. With this in mind, a rich text editor might be beyond what we need. \\

	\item
	\textbf{Book/chapter/scrap view} \\
	One of the elements of feedback received from our client is to make the user interface
	simpler for the user, so being able to see what chapters are listed under a book, 
	and what scraps are listed under each given chapter. Depending on how we choose to implement 
	this, it might prove difficult to have something that is appealing to our users and efficient
	for our application.\\


\end{enumerate}

	\begin{figure}[ht!]
	\centering
	\includegraphics[width=140mm]{prototype.png}
	\caption{Here is one low fidelity prototype that was shown to our client. Dr. Jensen suggested 
	removing the preview of chapters and scraps to make the interface a little more simple.}
	\end{figure}	


\section{Evan Term Progress}

\subsection{Weekly Recapping}
\noindent For the first week of the term, we were mostly catching up on what we had done
over the break. Steven and Josh had managed to meet with Carlos, which was great
to arrange future meetings and to get his opinion on where we were going with
the project. After a term of lower levels of contact than would have been
preferable, it was good to get this initial meeting set up and done with no
hassles. \\

\noindent Over the break, I began work on the book management backend library. This is the
foundation of the entire project -- the part that will do the actual work
interacting between the git repositories and JavaScript objects with convenient
APIs. In the second week, we got meetings set up with Jon, our TA. In the mean time I
wrote a basic preliminary version of the API specification for the library.
Barring any issues I ran into the plan was to implement this for the next
several weeks. \\

\noindent The third week was to be our first meeting with Dr. Carlos Jensen.
Unfortunately, due to a calendar migration relating to his change in job
position, he had missed our appointment scheduling and overbooked. We were
luckily able to work out a time to meet the next week with him. Week four was the first 
week where we were able to meet with Dr. Jensen as a
group. We go to talk about the state of the backend, including its technical
document storage design, and the frontend. For the frontend, Dr. Jensen asked
us to draw some rough prototypes. Steven is planning on taking the lead with
that.\\

\noindent We were unable to meet with Dr. Jensen during week five, as with his calendar
migration reset we changed our meetings to every other week. Instead, we used
that week to polish features we'd been working on already. For instance, I
finally got textbook rendering working. It was a pain to get, because it
involves descending into chapters and scraps (all asynchronous calls in
JavaScript), but with it working now that is a major milestone out of the way. 
At our next meeting, Monday of week six, we primarily went over our frontend
prototype designs with Dr. Jensen to get his feedback on user flow, interface
direction, and the like. We were able to refine our vision for the interface,
with his main feedback stressing making it simpler, and making it less
duplicative. We now have an idea around how the user will go about editing a
scrap, editing  a chapter, etc., and when in the writing process to start
rendering PDFs for layout preview.\\

\noindent This week, week six, has been all about capstone assignments -- the revised
documents, the video summary, and of course this progress report document.

\newpage
\subsection{What's Left on the Backend}
The backend is split into two primary parts. The first, and most difficult part
of the backend, is the interface library being used to store objects into the
git repositories and to get them back out. Shown below is a code sample, which
is fully working today, showing a simple script that creates new scraps,
chapters, and books, and gets a valid tex document to render from them.


\begin{lstlisting}
var a = new scrap.Scrap('First sentence of a chapter', 'rambourg');
await a.save('initial save');

var b = new scrap.Scrap('Second sentence of a chapter', 'rambourg');
await b.save('initial save');

var ch1 = new chapter.Chapter('First Chapter Name', 'rambourg');
ch1.addScrap(a);
ch1.addScrap(b);

await ch1.save('initial save');

var c = new scrap.Scrap('First sentence of another chapter', 'rambourg');
await c.save('initial save');
var ch2 = new chapter.Chapter('Another Chapter Title', 'rambourg');
ch2.addScrap(c);
await ch2.save('initial save');

var myBook = new Book('My Fav Book Title', 'rambourg');
myBook.addChapter(ch1);
myBook.addChapter(ch2);

var bookText = await myBook.getText();
\end{lstlisting}

\noindent We can create a new scrap with the scrap.Scrap constructor. We simply have
to pass it the text to include in the scrap, and the author. This allows us to
simply have some sort of API call that provides these two things and pushes them
straight into a scrap. It's necessary to save an object every time. After
creating the scraps we want, it's then time to create a chapter. It's worth
noting that normally, a user would more likely create a book first, then a
chapter, then some scraps, but since scraps, chapters, and books can exist
freely from each other -- a scrap doesn't always need to be associated with a
chapter, for example -- it's easiest to demonstrate here by creating them in the
order we associate them in. \\

\noindent After creating the chapter and saving it, we can
another create chapter as wanted, a new scrap, et cetera. We can then
just add these to a book by calling the addChapter method. When we're done and
want to get the text of the book in its entirety, we call the getText method.
What this method does is, more or less, recursively read all chapters, and then
all of their scraps, and then reconstruct the final latex file by concatenating
the values of the scraps.\\
\newpage
\noindent Let's take a closer look at the scrap object.

\begin{lstlisting}
var scr = new scrap.Scrap('Brody', 'abc123-def456')

/* reconstitute a chapter from author and id */
var ch = chapter.reconstitute('Patrick Rambourg', '56c2c2ee')
var newCh = await ch.fork('Brody')

newCh.addScrap(scr)
await newCh.save('add new chapter')
\end{lstlisting} 

\medskip
\noindent We have today a working
implementation to load chapters, scraps, and books from disk, modify them, and
save them. As shown here, it's as easy as calling "reconstitute" with the
author's name and the object's ID number. In the future, another user will be
able to fork an object, which will create a copy of the object in their space
that they can modify. They'll then be able to do things like add scraps to a
chapter, chapters to a book, modify scraps, et cetera. It's worth noting that
forking is not currently implemented. However, we do not anticipate any problems
with its implementation, as it integrates easily into the existing git backend. \\

\noindent So what is currently left on the backend? As mentioned before, we have 
two main parts. The library is the unique part of the project, the part that controls
the actual object storage, modification, forking, and saving. Of these
functionalities, the only parts left are forking, and getting previous versions.
Getting a previous version is around 80% done. We can currently read the object
from the disk and retrieve the old version. It is now only a matter of creating
a new object from this old version and returning it.\\

\noindent The other main part of the backend is the API. This API is a simple app written
in Node that wraps the backend library in web-available JSON requests. The API
is going to be the easier of the two pars to implement, as I already have ample
experience creating web APIs, including in NodeJS. The API will include user
authentication, allowing us to control who modifies which objects, and keep
track of the owner.\\

\noindent In addition, we will be needing some kind of search backend. We have yet to
develop our exact strategy for search, but will most likely be using a prebuilt
search backend and simply loading objects into it at save time. This will allow
for easy referencing back and forth between search results, which will return
the object id and its owner, and the object stored on disk, which is referred to
by its id and owner.\\
\newpage
\section{Josh Term Progress}
\subsection{Weekly Recapping}
\noindent Towards the very end of winter break, we were able to get a meeting with Carlos to clear
up some possible fallout from the previous term. While we were able to get
confirmation on some of the technologies we were using, we weren't able to sort out some
questions on the backend because Evan is the main group member handling that side of the
application. We were able to, however, figure out that we could meet with
him again the following week after he moved into his new office. This meant we weren't
able to meet with him for at least two weeks, however. \\

\noindent After our meeting, we decided to delegate some time to developing a game plan of how we
could divide and conquer the majority of it. The following week, our whole group 
managed to meet and accomplish setting up the basic architecture blueprints for the 
application. This meant that Steven and I would forward with Aurelia, and do 
everything we could get a working version up and running as soon as possible.\\

\noindent Our client ended up needing more time to delegate to other task on hand, and this meant
switching our meeting frequency to a biweekly basis. This hasn't caused for any problems,
and actually allows for us to bring more deliverable content to our meetings. Two week
springs tend to be more ideal in that regard, however, we'll be working a little less
close with our client in the future because of this.\\

\noindent While the backend had been progressing with Evan, Steven and I focused our time 
trying for a semi working version of a local Aurelia project that would run in browser. 
It wasn't too difficult to follow an example online and get an express version of the 
application up and running. Our group was then able to meet that week to briefly discuss 
progress and impediments. \\

\noindent During our 14th week, we finally managed to make some real progress and gain momentum, as
we were able to meet with Dr. Jensen. This led to discussion about work flow designs with a
desired low fidelity goal. This outlined adequate deliverables for our next meeting two weeks
from then, and gave us expectations as to what our client wanted to see from us. During the meeting, 
Steven and I managed to demo a fully working version of a basic Aurelia project to Dr. Jensen. 
This was useful to to keep him updated with our current progress. After the meeting, we moved 
forward with the goal of having a few basic layouts prepared, as well
as some further technology integration.\\

\noindent Steven and I started to work on a PDF viewer that we could integrate into our application,
and managed to find an online guide that would walk us through setting one up with Aurelia
as our framework. While we managed to get a PDF viewer running on a separate instance, Steven
and I learned quite a lot from the experience and furthered our knowledge on working with
Aurelia in general. We had our meeting with Dr. Jensen later this week, and we wanted to have
this accomplished going into our meeting. From this point, Steven worked on prototypes for
the layout design and I branched off into finishing the pdf viewer demo. \\

\noindent Our next meeting with Dr. Jensen went very well, he gave us constructive feedback on 
our low fidelity prototype designs and we were able to demo the front of the application to him.
Our group also started work towards our progress report, Which will slightly hinder the time
dedicated to pushing our application forward.\\

\subsection{Overview}
\noindent The first real development our team started to work on was the user interface. We have already
dedicated a fair amount of time to this section, as the user interface is one of the most
integral parts of our application, and to any application for that matter. Studies have shown
that the average user will stay on a new website for as little as one minute, which means
that making the user interface appealing to the user as well as easy to use is crucial to the
overall success.\\

\noindent Our goal was to start on this a couple of weeks before winter term started, that way we could
get a good lead and gain momentum. However, previous impediments had led us to wait on this
aspect, this way, however, we could get insight into what our client had envisioned for this
part of the application. We're planning to be working on this for around 7 weeks total, which
should give us plenty of time to move forward on this major part of the project. More than
just these 7 weeks, we'll be adding continuously to the U.I. as our application continues to
grow.

\subsection{What's Left on the Frontend}

\noindent  For this next section, I'll briefly go over some of the future implementations 
and tweaks that I'll be taking on personally or with the help of Steven.

\begin{enumerate}
    \item
    \textbf{Integration of PDF Viewer and Application} \\
	We currently have a working instance of an Aurelia project with a basic front page layout,
	as well as a different project instance that contains the PDF viewer. While both of these
	seem to work well on their own, integrating them together will take some time and massaging
	in order to obtain the desired functionality. The pdfviewer is an essential part of the
	application, which means that this needs to be integrated in such a way that it won't cause
	difficulty when trying to work with it later on. \\

    \item
    \textbf{Finalizing layout design} \\
	Last week, we went into our client meeting with a few layout prototypes, and with this we
	were able to narrow down what was expected. We have the basic layout, and our next still
	will be taking it from a low/mid level fidelity prototype to a high fidelity layout. \\

	\item
	\textbf{Working scrap editor} \\
	This section will take a little bit more work, as the combination between backend and
	frontend will merge in some aspects. On the frontend, this won't be such a challenge.
	We need to either find an open source rich text editor/ or build one from scratch.
	Making one from scratch would be more time consuming, but we would have more control
	and knowledge as far as how it works. This will look like a word document or any other
	basic text editor or formatter. Seeing as this is one of the more important aspects of
	our application, we're approximating it to take around 4 weeks. If it ends up taking more
	than this, we'll delegate more time appropriately. \\

	\item
	\textbf{Book/chapter/scrap view} \\
	This part might be more complicated. Seeing as we're a single page application, we might
	have to look into reformatting a good amount of the front end design. Implementing a router
	would be the only way to ensure multiple page views, but shouldn't cause too much drawback. \\


\end{enumerate}


\section{Current Project Status}
\noindent Steven:

\noindent Currently our project is a little behind schedule. We are attempting to make up
lost time by meeting more frequently and having sprints for given tasks. We have updated our 
gantt chart that has moved up various tasks so that if we take longer on a given task that 
it will have started sooner as well. We have revised all of our written documentation, reflecting
any changes that we have made after they were written. We are still partially in the planning 
phase for both the frontend and backend systems and continue to iterate on ideas as we progress
through the term. \\

\noindent Our next few goals are to have an integrated PDF viewer within our main Aurelia 
application, backend and frontend integration so we can begin working on sending and receiving 
information, and making more progress with User Testing and associated prototypes. \\

\noindent We will continue with weekly meetups, work individually, or in groups so we will 
be able to progress even when one individual team member is blocked on either something 
relating to the project, or on other work.



\section{Conclusion}
\noindent Steven:

\noindent The Many Voices Publishing Platform has been a great project to work on,
bringing each team member outside their comfort zone. A lot of planning had
taken place over the previous term, sometimes resulting in shifts in direction of how
to manage and develop the platform. Now that we are in Winter Term, the Remix team feels 
more comfortable as we develop the platform. Part of this comfort comes from the the biweekly our client, 
and weekly meetings with our TA Jon Dodge who is providing us guidance as we move onto different
project tasks and class requirements.

\end{document}
