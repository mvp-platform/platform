\documentclass[onecolumn, draftclsnofoot,10pt, compsoc]{IEEEtran}
\usepackage{graphicx}
\usepackage{url}
\usepackage{setspace}
\usepackage{comment}
\usepackage{bigstrut}
\usepackage{geometry}
\usepackage{tabu}

\geometry{textheight=9.5in, textwidth=7in}

% 1. Fill in these details
\def \CapstoneTeamName{		Remix}
\def \CapstoneTeamNumber{		61}
\def \GroupMemberOne{			Josh Matteson}
\def \GroupMemberTwo{			Steven Powers}
\def \GroupMemberThree{			Evan Tschuy}
\def \CapstoneProjectName{		Many Voices Platform}
\def \CapstoneSponsorCompany{	Oregon State University}
\def \CapstoneSponsorPerson{		Carlos Jensen}

% 2. Uncomment the appropriate line below so that the document type works
\def \DocType{
        %Problem Statement
				%Requirements Document
				%Technology Review
				%Design Document
				Progress Report
				}

\newcommand{\NameSigPair}[1]{\par
\makebox[2.75in][r]{#1} \hfil 	\makebox[3.25in]{\makebox[2.25in]{\hrulefill} \hfill		\makebox[.75in]{\hrulefill}}
\par\vspace{-12pt} \textit{\tiny\noindent
\makebox[2.75in]{} \hfil		\makebox[3.25in]{\makebox[2.25in][r]{Signature} \hfill	\makebox[.75in][r]{Date}}}}
% 3. If the document is not to be signed, uncomment the RENEWcommand below
\renewcommand{\NameSigPair}[1]{#1}

%%%%%%%%%%%%%%%%%%%%%%%%%%%%%%%%%%%%%%%
\begin{document}
\begin{titlepage}
    \pagenumbering{gobble}
    \begin{singlespace}
    	\includegraphics[height=4cm]{coe_v_spot1}
        \hfill
        % 4. If you have a logo, use this includegraphics command to put it on the coversheet.
        %\includegraphics[height=4cm]{CompanyLogo}
        \par\vspace{.2in}
        \centering
        \scshape{
            \huge CS Capstone \DocType \par
            {\large\today}\par
            \vspace{.5in}
            \textbf{\Huge\CapstoneProjectName}\par
            \vfill
            {\large Prepared for}\par
            \Huge \CapstoneSponsorCompany\par
            \vspace{5pt}
            {\Large\NameSigPair{\CapstoneSponsorPerson}\par}
            {\large Prepared by }\par
            Group\CapstoneTeamNumber\par
            % 5. comment out the line below this one if you do not wish to name your team
            \CapstoneTeamName\par
            \vspace{5pt}
            {\Large
                \NameSigPair{\GroupMemberOne}\par
                \NameSigPair{\GroupMemberTwo}\par
                \NameSigPair{\GroupMemberThree}\par
            }
            \vspace{20pt}
        }
        \begin{abstract}
        % 6. Fill in your abstract
		\noindent This document summaries the progress that the Remix team 
		has made on the Many Voices Publishing Platform for the client 
		Dr. Carlos Jensen. Additionally this document provides a week by 
		week summary of work performed, as well as what is needed to 
		be changed to improve effectiveness in building the MVP platform.
        \end{abstract}
    \end{singlespace}
\end{titlepage}
\newpage
\pagenumbering{arabic}
\tableofcontents
% 7. uncomment this (if applicable). Consider adding a page break.
%\listoffigures
%\listoftables
\clearpage


% 8. now you write!
\section{Project Purpose}
\noindent A modern textbook is updated frequently, perhaps even yearly, 
and can cost in the range of hundreds of dollars. Students are often left 
to attempt to understand poorly worded, even incorrect information from a textbook 
often chosen from those sent to a professor for review by the publisher. 
This can lead to better works with less aggressive sales tactics not being 
made available, or even known. 
Another choice would be for a professor to write their own textbook. 
However, this is a process that takes months of endless research and 
time spent, and on top of that will require peer review and publishing 
before it can be released. \\

\noindent The Many Voices platform offers to put an end to this massive, 
slow, expensive cycle.  Instead of a textbook being a single document 
written by one professor, we seek to re-imagine the textbook as instead 
a collection of content written by professors from around the world 
that are useful for a particular class. A knowledgeable professor can 
contribute a few chapters on their specialty, without needing to write 
an entire textbook around it. \\

\noindent Professors wishing to use this content can then modify it for 
their uses in the classroom. The material will be hosted in such a way 
as to provide the ability to \"fork\" content, or create content based 
off of it. The platform will provide a way to search for and find content, 
prioritized by relevance and credibility as determined by other users; 
the most popular material will be shown with the most prominence. 

\section{Weekly Updates}

\subsection{Week 3}
The term started with ironing out the problem statement. At the end of the 
second week, we met with our client, Dr. Jensen, and briefly went over exactly 
what his vision was for the end result of the platform. The third week was 
then spent on the initial drafting of the problem statement, which was turned 
in at the end of the week. Unfortunately, we were unable to meet with Carlos 
again to go over our draft, as he was unavailable.

\subsection{Week 4}
We attempted to get Carlos' feedback on our problem statement document, 
and get him involved in the creation of the coming requirements document, 
but were unfortunately again unable to find a meeting time that worked 
well for him. Therefore, we simply pressed on and began the revision process 
using the feedback our TA, Jon Dodge. We spent some of the time investigating 
tools that would possibly be useful in the development phase of the product.

\subsection{Week 5}
This week, we managed to get a hold of our client and get him to sign our 
revised Problem Statement. We previously were having quite a bit of trouble 
getting a hold of him because he's been traveling. We also managed to all 
work together on our Requirements Document, which has been one of the few 
times we've all been able to find times in our schedule to do it together 
(even though it was over the internet). The Requirements document was 
pretty intimidating looking, simply by virtue of its size!

\subsection{Week 6}
During week six, Steven made a lot of progress revising the requirements document, 
and then integrated the changes Evan and Josh made to the document. He was also able 
to attend Dr. Winters' writing session, where we received useful and actionable 
feedback about documentation formatting. We also revised the Problem Statement 
after hearing from Dr. Winters that we may be able to re-submit revised documents 
for regrading. This will hopefully prevent us from receiving another 82/100. 
Two drafts were sent out to our client, with 48 hours notice each time, but 
unfortunately was only able to get a signature on the second a few hours before 
turn-in time. The main issue we had this week were slight differences in the 
requirements document formatting compared to the IEEE 830 format specification, 
though our TA Jon Dodge and Dr. Winters feel that the document looks great. 
It was also slightly concerning that we were not able to get feedback apart 
from the signature of our client, though this is understandable because he is 
traveling right now.


\subsection{Week 7 \& 8}
The technology review document was the main point of focus during Week 7, with 
it being due at end-of-day Monday of Week 8. Steven also made revisions to the 
requirements document, including many suggested by Jon Dodge, to prepare for a 
re-submission for regrading. Evan was not available for a good portion of the 
week seven, as he was on an important trip to California, but as we split the 
document into its constituent sections early in the week, he was able to work 
on his way down. On Monday, we combined the efforts that had been written up 
to that point. Unfortunately, we were not satisfied with the state of the 
document, and so we requested an extension. We were, along with the rest of 
the class, given an additional 36 hours, pushing the due date back to Wednesday 
at noon, when we turned in a satisfactory document.


\subsection{Week 9 \& 10}
Week 9 was mainly spent with our respective families, and as a short break from 
the march towards project completion. We made the strategic decision to focus 
on getting rested and well-fed in an attempt to mentally prepare ourselves for 
the last two main parts of the Fall-term requirements. \\
 
\noindent Week 10 began with work towards finishing our design document. 
After a few false starts and a night of less sleep than desired, the final 
design document was turned in on time. Unfortunately, we were unable to get 
our client's signature in time; we will be providing a signed copy to be graded 
as soon as possible. Later in Week 10, during the weekend before finals week, 
we wrote our progress update document and recorded our progress update video.


\section{Steven Term Progress}
For this week we were getting back into the groove of things after Winter break. 
Josh and I were able to meet with Carlos the week prior and talked about 
arranging meetings going forward each week.

We weren't able to get a whole lot of work done this week, but we planned for 
meeting each week at least a few hours a couple days of the week so we can 
continue making progress. We plan to meet next week and work on the framework 
and user stories.

This week we began our weekly meetings, though our client had to cancel 
meeting for the first two weeks. We also setup our meetings with Jon, which was 
actually pretty difficult to find a time that worked for all four of our schedules.

We plan to continue working on the User Stories and Framework so we are able 
to make progress towards a working prototype.

This week we were to begin our weekly meetings with our client. 
Our client recently transferred positions and our weekly appointment was 
discarded. This resulted in our weekly time slot being given away, so now we 
have to find a new time (planned for 9:00AM Mondays). Our client also wants 
to only meet every other week rather than weekly now.

We talked over the past terms documents and possible upcoming documents that 
we would be seeing in the class and that we should be focusing on development 
throughout this term.

We worked on the backend system and testing of the include and input commands 
with our setup (which seems to be working well!). We also looked into security 
practices to protect our LaTeX documents from succumbing to common exploits 
(escaping the document and accessing the shell).

We also worked on getting the framework up and running (TS Lint has been giving 
us some issues, so we might go with straight typescript without LINT and go to 
JavaScript with Lint.

This week we met with Dr. Jensen and discussed the back-end of the system as 
well as some plans for how to model the front end of the system.

Dr. Jensen suggested using low fidelity / medium fidelity to receive feedback 
from users instead of focusing on getting to high fidelity and having users 
review that as they will be less critical of the whole system and instead 
focus on pixel alignment.

Josh and I need to get more work done on Aurelia and get a basic user interface 
up and running and try working towards in page rendering of the resulting 
PDF from the back end system.


This week was a week we had off from meeting with Dr. Jensen, we used this time 
to work on developing some of the required features that we need. This 
included Josh and I working on the PDF in browser rendering with Aurelia. 
Josh finished this task up separately.

I worked on Prototype development to have Dr. Jensen review and provide 
feedback on during next weeks meeting. I wasn't able to get as many prototypes 
drawn as I would have liked, but there is still a good chunk of design that 
Dr. Jensen could critique.

Next week I plan to show Dr. Jensen our designs and take in any feedback he 
might have. I also plan to work on revising all of our documents and get them 
placed into our OneNote document that I have prepared for our group.


This week we met with Dr. Jensen and he had lots of feedback about our designs, 
mainly in ways to simplify the interface for the users while also making it 
easier to develop. We talked about rendering the PDF only when needed 
(as in active editing) instead of constantly. We also talked about the 
possibility of using a tabbed system to only render when the user wants to, 
which will reduce our server calls.

I plan to continue working on our document revisions, as well as our Progress 
Report materials (video, presentation materials, and report), which will mainly 
summarize the previous terms work and provide an overview of our progress 
this term.

Something that will be better next week is availability, as this week has a 
lot of midterms for various classes.


\section{Evan Term Progress}

This week, we're not really working on too much. We'll be meeting with Carlos 
in about a week and a half, which will hopefully give us an opportunity to 
solidify what he wants, versus what we've been designing.

In the mean time, I worked on some of the book management backend over break. 
I'll continue working on that in my free time.


This week we set up meetings with Jon and I've fleshed out the required API 
for the backend. Barring issues, I'm going to be implementing that for the 
next few weeks.

This week Carlos cancelled due to calendar migration issues, so we've 
rescheduled for next week at 9am. Hopefully that'll work out...
I'm working on getting that backend still moving forward. 

This week we got to meet with Dr Jensen about the state of the backend system 
(which is my primary domain) and the front-end design, which Steven is planning 
on drawing some rough prototypes for.

My plan for the next few days is to work on more backend integration, 
including getting the textbook to render properly.

We didn't meet with Carlos this week, as we've made those every-other. 
Instead, we used the time to polish existing features we'd already worked on, 
and write new frontend and backend features. Personally, I got textbook 
rendering working, which was annoyingly difficult, but it's finally done!

At our next meeting, we're primarily going to show Carlos our frontend 
prototype designs to get feedback on user flow, interface direction, 
and the like.

After our meeting with Carlos, we were able to refine our vision for the 
interface. He mainly stressed making it simpler, and making it less duplicative. 
We now have an idea around how the user will go about editing a scrap, editing 
a chapter, etc., and when to render PDFs.

This week is mainly going to be capstone assignments -- the revised documents 
and the video summary.


\section{Josh Term Progress}
\subsection{Recapping}
Towards the very end of winter break, we were able to get a meeting with Carlos to clear 
up some possible fallout from the previous term. While we were able to get 
confirmation on some of the technologies we were using, we weren't able to sort out some 
questions on the backend because Evan is the main group member handling that side of the 
application. We were able to, however, figure out that we could meet with 
him again the following week after he moved into his new office. This meant we weren't 
able to meet with him for at least two weeks, however.

After our meeting, we decided to delegate some time to developing a game plan of how we 
could divide and conquer the majority of it.

The following week, our whole group managed to meet and accomplish setting up the basic
architecture blueprints for the application. This meant that Steven and I would forward 
with Aurelia, and do everything we could get a working version up and running as soon 
as possible.

Our client ended up needing more time to delegate to other task on hand, and this meant 
switching our meeting frequency to a biweekly basis. This hasn't caused for any problems,
and actually allows for us to bring more deliverable content to our meetings. Two week 
springs tend to be more ideal in that regard, however, we'll be working a little less 
close with our client in the future because of this. 

While the backend had been progressing with Evan, Steven and I focused our time trying for 
a semi working version of a local Aurelia project that would run in browser. It wasn't too
difficult to follow an example online and get an express version of the application up and 
running. Our group was then able to meet that week to briefly discuss progress and 
impediments. 

During our 14th week, we finally managed to make some real progress and gain momentum, as 
we were able to meet with Dr. Jensen. This led to discussion about work flow designs with a 
desired low fidelity goal. This outlined adequate deliverables for our next meeting two weeks 
from then, and gave us expectations as to what our client wanted to see from us. 

During the meeting, Steven and I managed to demo a fully working version of a basic Aurelia 
project to Dr. Jensen. This was useful to to keep him updated with our current progress. After 
the meeting, we moved forward with the goal of having a few basic layouts prepared, as well 
as some further technology integrations. 

Steven and I started to work on a PDF viewer that we could integrate into our application, 
and managed to find an online guide that would walk us through setting one up with Aurelia 
as our framework. While we managed to get a PDF viewer running on a separate instance, Steven 
and I learned quite a lot from the experience and furthered our knowledge on working with 
Aurelia in general. We had our meeting with Dr. Jensen later this week, and we wanted to have 
this accomplished going into our meeting. From this point, Steven worked on prototypes for 
the layout design and I branched off into finishing the pdf viewer demo. 

Our next meeting with Dr. Jensen went very well, he gave us constructive feedback on our low 
fidelity prototype designs and we were able to demo the front of the application to him. 
Our group also started work towards our progress report, Which will slightly hinder the time 
dedicated to pushing our application forward.

\subsection{Overview}
The first real development our team started to work on was the user interface. We have already 
dedicated a fair amount of time to this section, as the user interface is one of the most 
integral parts of our application, and to any application for that matter. Studies have shown 
that the average user will stay on a new website for as little as one minute, which means 
that making the user interface appealing to the user as well as easy to use is crucial to the 
overall success.

Our goal was to start on this a couple of weeks before winter term started, that way we could 
get a good lead and gain momentum. However, previous impediments had led us to wait on this 
aspect, this way, however, we could get insight into what our client had envisioned for this 
part of the application. We’re planning to be working on this for around 7 weeks total, which 
should give us plenty of time to move forward on this major part of the project. More than 
just these 7 weeks, we’ll be adding continuously to the U.I. as our application continues to 
grow.

\subsection{What's Left on the Frontend}

For this next section, I'll briefly go over some of the future implementations and tweaks that
I'll be taking on personally or with the help of Steven.

\begin{enumerate}
    \item
    \textbf{Integration of Pdf Viewer and Application} \\
	We currently have a working instance of an Aurelia project with a basic front page layout, 
	as well as a different project instance that contains the PDF viewer. While both of these
	seem to work well on their own, integrating them together will take some time and massaging 
	in order to obtain the desired functionality. The pdfviewer is an essential part of the 
	application, which means that this needs to be integrated in such a way that it won't cause
	difficulty when trying to work with it later on. \\

    \item
    \textbf{Finalizing layout design} \\

	Last week, we went into our client meeting with a few layout prototypes, and with this we 
	were able to narrow down what was expected. We have the basic layout, and our next still 
	will be taking it from a low/mid level fidelity prototype to a high fidelity layout. \\

	\item
	\textbf{Working scrap editor} \\

	This section will take a little bit more work, as the combination between backend and 
	frontend will merge in some aspects. On the frontend, this won't be such a challenge. 
	We need to either find an open source rich text editor/ or build one from scratch. 
	Making one from scratch would be more time consuming, but we would have more control
	and knowledge as far as how it works. This will look like a word document or any other 
	basic text editor or formatter. Seeing as this is one of the more important aspects of 
	our application, we’re approximating it to take around 4 weeks. If it ends up taking more 
	than this, we’ll delegate more time appropriately. \\

	\item
	\textbf{Book/chapter/scrap view} \\

	This part might be more complicated. Seeing as we're a single page application, we might 
	have to look into reformatting a good amount of the front end design. Implementing a router 
	would be the only way to ensure multiple page views, but shouldn't cause too much drawback. \\


\end{enumerate}

\section{Fall Term Weekly Summary}

\begin{tabular}{|p{0.05\linewidth}|p{0.285\linewidth}|p{0.285\linewidth}|p{0.285\linewidth}|}
\hline 
Week & Positives & Deltas & Actions \\ \hline

	3 
	& Got the initial draft of the problem statement done after meeting last 
	week with our client. 
	& Were unable to set up a meeting with our client, and thus had no way of 
	getting feedback on the draft before submission. 
	& We will talk to our client via email about possible meeting times, either 
	in person or remote, that may possibly work for everyone. \\ \hline

	4 
	& Figured out a time for meeting with our TA. 
	& Need to hear back from our client. 
	& We will wait another day or two and then send a new polite email. \\ \hline

	5 
	& Finished revising the problem statement, got it signed. 
		Began the requirements document. 
	& N/A 
	& N/A\\ \hline

	6 
	& Revised Requirements document via feedback from TA. Got client signature on Requirements. 
		Attended writing workshop. 
	& Need to be on top of things regarding contacting our client as 
		he is difficult to contact. 
	& We will send him another email asking for feedback on the requirements. \\ \hline

	7 
	& Began technology review. Revised requirements document for possible re-grading. 
	& Need better team work schedule to get times to work on documents 
	together rather than separately. 
	& We will discuss possible extra meeting times in addition to our weekly meeting 
	with our TA where we can work for several hours uninterrupted. \\ \hline

	8 
	& Finished technology review (after getting extension to Wednesday). 
	& Need to be on top of documents to finish them well in advance of due dates! 
	& We will personally push ourselves to get work done sooner rather than later. \\ \hline


	9 \hspace{3mm} \& 10 
	& Spent Thanksgiving Week getting rested. Finished design document. 
		Finished progress update document/finished presentation. 
	& Need to get client signature for design document! 
	& We will be emailing Carlos. With the new term we hope 
		to set up a weekly meeting with him.  \\ \hline

\end{tabular}


\section{Fall Term Impending Problems}

For Fall term, client communication was a problem that impeded our progress 
at a few points. Due to busy schedules of the team and the client, communication 
slipped from where we expected it to be.
This caused problems with project requirements and other questions the team 
had about moving forward. For Winter term and beyond, the team plans on having 
a weekly meeting with our client and additionally providing a weekly email 
detailing our progress of the week. \\ 

\noindent Additionally, finding time for the team to come together to work 
on the project proved difficult.
Our schedules had many conflicting times with classes and work times that 
made it difficult to spend large portions of time together.
This resulted in a lot of remote development of the planning and documentation, 
resulting in less detailed documentation and lower scores on grading.
In the recent weeks, the team has spent more time working together, which has 
led to more cohesive development. For Winter term and beyond, the team has 
schedules that align more cohesively, allowing for more time to be spent 
discussing and developing the platform together.

\section{Retrospective}

\begin{tabular}{|p{0.3\linewidth}|p{0.3\linewidth}|p{0.3\linewidth}|}
\hline

	Positives 
	& Deltas 
	& Actions \\ \hline

	Team came together on planning and design 
	& Client communication 
	& We will talk to our client via email about possible meeting times, 
		either in person or remote, that may possibly work for everyone. \\ \hline

	Learned a lot about the software development process 
	& Documentation / Development Confusion 
	& If we are confused and blocked by something in the class that could 
		be helped by asking a question of either our TA Jon Dodge or Professors 
		D. Kevin McGrath or Dr. Kirsten M. Winters, we will. \\ \hline

	Learned a lot about Latex and the writing of technical documents. 
	& Team Communication 
	& Solved: Problems with communication were solved by 
		transition to Slack \& Email communication \\  \hline

	& Team Meeting Time 
	& Need to compare our schedules and find a time that will work 
		for all of us to get together. \\ \hline

\end{tabular}

\section{Current Project Status}

Thus far we have written documentation detailing the technical and design 
requirements of the project.
Now, we are beginning to move into the technical development phase.
To begin, we are finishing coalescing a unified vision of what the project is, 
and how we will go about architecting and building it.
We are planning on beginning to build our initial prototype over winter break, 
based on the basic skeleton laid out in the documentation. \\

\noindent Moving forward, we expect to spend the majority of time doing 
individual development work, with weekly team development sessions to 
keep ourselves on the same track.
In doing so we will be able to progress even when one individual team 
member is blocked on either something relating to the project, or on other work.
To do this, we have split the project into chunks, which we will then 
work on either solo or in a pair. 



\section{Conclusion}

The Many Voices Publishing Platform has been a great project to work on, 
bringing each team member outside their comfort zone. A lot of planning has 
taken place over Fall term, sometimes resulting in shifts in direction of how 
to manage and develop the platform. The Remix team feels more comfortable moving 
into Winter term, and for development to begin on the platform. Part of this 
comfort comes from the planning of weekly meetings with the client, and weekly 
emails to detail the current project status.

\end{document}
