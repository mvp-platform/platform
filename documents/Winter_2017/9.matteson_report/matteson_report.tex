\documentclass[onecolumn, draftclsnofoot,10pt, compsoc]{IEEEtran}
\usepackage{graphicx}
\usepackage{setspace}
\usepackage{comment}
\usepackage{bigstrut}
\usepackage{geometry}
\usepackage{supertabular}
\usepackage{tabu}
\usepackage{hyperref}
\usepackage{url}
\usepackage{afterpage}

\geometry{textheight=9.5in, textwidth=7in}

\begin{document}

\noindent Josh Matteson

\noindent Winter Final Progress Report

\noindent Group 61

\noindent The Many Voices Publishing Platform


\section{Introduction}

The purpose of our application is to make an easy to use workflow for creating a book or working 
off somebody else's book. The textbook market is crowded with over priced, misleading, and sometimes 
inaccurate information filled textbooks. To fix this problem, our application will help professors 
make textsbook in exactly the fashion that they desire. Whether that's entirely different from the 
one they started with, or just missing a few chapters from the original. The preference will be up 
to the user, but ultimately, what we want to accomplish with our application is to give everyone 
a simple tool that they can use to easily create a textbook specific enough for their liking. \\

\section{My Work}

The first piece of technology that I am responsible for is the testing portion. We are going to be using karma and the basic testing 
suite that comes with Aurelia. Using this technology will aid in proper functionality and minimize errors. Without properly testing 
code, a number of problems can occur that can disrupt and slow down progress in a team. In extreme cases, not properly testing could 
lead to failure of the application. \\

The second technology I am responsible for is the revision control unit. While this is my technology, this has fallen under the work 
of the backend. This part of the application is being delegated to Evan. It's designed to handle small to very large projects with speed and accuracy.
It is commonly used by a variety of companies. \\

Lastly the database technology. We chose This particular part of the project revolves around how easy it is to use the calls to the database, and how 
how easy it is to incorporate it into the application. We chose to go with SQL Server to accomplish this goal. \\



\section{Progress and Problems}
Towards the very end of winter break, we were able to get a meeting with Carlos to clear
up some possible fallout from the previous term. We were able to figure out that we could meet with
him again the following week after he moved into his new office. After this point, we were able to 
move past our previous issues of having poor communication with our client. We established that meetings 
would happen every other week in order to keep in touch with the project as well as have him assist us 
wherever possible. \\

After our meeting, we decided to delegate some time to developing a game plan of how we
could divide and conquer the majority of it. The following week, our whole group 
managed to meet and accomplish setting up the basic architecture blueprints for the 
application. This meant that Steven and I would forward with Aurelia, and do 
everything we could get a working version up and running as soon as possible. \\

While the backend was progressing with Evan back in the first weeks, Steven and I focused our time 
trying for a semi working version of a local Aurelia project that would run in browser. 
It wasn't too difficult to follow an example online and get an express version of the 
application up and running. Our group was then able to meet that week to briefly discuss 
progress and impediments. \\

During our 14th week, we finally managed to make some real progress and gain momentum, as
we were able to meet with Dr. Jensen and talk about what we needed from him. This led to discussion 
about work flow designs with a desired low fidelity goal. This outlined adequate deliverables for 
our next meeting two weeks from then, and gave us expectations as to what our client wanted to 
see from us. During the meeting, Steven and I managed to demo a fully working version of a basic 
Aurelia project to Dr. Jensen. This was useful to to keep him updated with our current progress. 
After the meeting, we moved forward with the goal of having a few basic layouts prepared, as well
as some further technology integration.\\

One of the most important meetings took place after this, where we were able to hammer out 
all our user stories and figure out what we needed for both the front end and the backend. The next 
meeting we had with Dr. Jensen went really well, as we had a lot of deliverable content. We were able 
to show Dr. Jensen all our layout prototypes, and overall he seemed quite happy with the direction in 
which we were taking the layout. \\

The next meeting our team had was groundbreaking as we were able to meet up and completely 
cover the API calls and backend requirements, which caused us to be set up really well for moving 
forward. Steven and I now understand what is needed to accomplish all our goals, which will help us 
tremendously when creating the user interface. \\

Steven and I were able to get a good amount of work done covering the front end. We worked 
on implementing the router and tried to get basic tab functionality working. Unfortunately we weren't
 able to get the tabs and has been a smaller impediment. However, we were still able to finally get the
 router working, which helps the user navigate from page to page of the application without refreshing. 
 Carlos asked me to start thinking critically about an algorithm for compiling latex into a PDF in a 
 calculated time. I'm going to start thinking and doing research on that topic before our next meeting 
 with Carlos come spring term. \\

 As far as impediments that have come up this term, I believe one of the biggest ones was not 
 being able to delegate enough time to working on the application. We still have so much to do before 
 expo finally arrives, and this means that quite a lot of work will have to be done over spring break. 
 Steven and I have been able to accomplish quite a lot in the time we've been given, and so I'm not too 
 worried about us being able to meet all of our goals on time. Below is an illustration of what all the 
 progress Steven and I have made on the frontend currently looks like. \\

\includegraphics[scale=0.3]{current_frontend}

\section{Member Evaluations}

\subsection{Steven Powers}
Overall, I would say that Steven has shown serious dedication in the way that he has helped our senior capstone project move forward. 
Evan and I both consider Steven to be the manager of the group, and has been the main one in contact with Dr. Jensen. He makes sure 
that all of our projects have been turned in our time, and does a majority of the filler work for our papers and smaller projects. \\

Steven also helps me as much as he can with the frontend of the application. He hasn't had any real experience working with a Javascript 
framework before, but he is very attentive and tries to help as much as he can with it. \\

\subsection{Evan Tschuy}
Evan has been the one who is solely in charge of our backend. He alone has ensured that the quality and functionality is meeting 
our goals and expectations. Evan also contributes to our meetings by ensuring that we talk about everything we need to talk about, 
and ensuring that we're accomplishing all of our goals. \\

\subsection{Myself}
I would say that the biggest factor I contribute to the team is organizing time for us to meet up, and the frontend of our application. 
On my previous internship, I worked heavily with another Javascript framework like the one we're using now. This lead to relative ease
when trying to understand how to work with the new framework we were all going to have to use. Steven has helped me out a great deal with 
research and useful articles, however, I would say the majority of the frontend has been my work and research. This will change as he 
becomes more familiar with how to work with a Javascript framework and we both can be working on the frontend together.  \\

\subsection{Overall}
Everyone in the team seems to have their own specific role and equal level of contribution. We all spend about an equal amount of time in 
different ways adding to the project. Everyone seems to be serious about doing well, and the only part I would say we all need work on is 
setting apart more time to accomplish everything we need. \\
\afterpage{\clearpage}
\section{Retrospective}
\begin{tabular}{|p{0.05\linewidth}|p{0.285\linewidth}|p{0.285\linewidth}|p{0.285\linewidth}|}
\hline
	Week & Positives & Deltas & Actions \\ \hline

	1
	& Cleared up fallout with poor communication between us and client.
	& Evan wasn't able to make it to the first meeting, which made for some unanswered questions during it.
	& Steven and I were able to talk with Evan about what he missed and sorted out what he needed for the next meeting. \\ \hline

	2
	& Our group was able to meet up and discuss the basic architecture of the application and how we should move forward.
	& Our client wasn't able to meet with us and cancelled our meeting.
	& We talked with our TA (Jon Dodge) about what we should do about meetings and communication moving forward. \\ \hline

	3
	& Steven and I managed to get a semi working frontend that we could show, and Evan continued to work on the backend.
	& Our client began migrating to a new job and wasn't able to meet with us again because of this.
	& We were finally able to talk with him and agree upon a bi-weekly meeting time, which will be cause for more deliverable content each week  \\ \hline

	4
	& Discussed important ways to progress with Dr. Jensen, which included low fidelity prototypes.
	& Haven't been consistent with our meeting times.
	& Discussed important of these topics during the meetings, which aided in the meetings becoming more important.  \\ \hline

	5
	& Finished plugin which will allow us to view PDF's inside our app.
	& Personally need to integrate the PDF Viewer with the rest of our application.
	& Set a part enough time to integrate the PDF Viewer into the application. \\ \hline

	6
	& Met with Dr. Jensen with plenty of deliverable content, including the integration of the PDF Viewer and several low fidelity prototypes to show.
	& Lots of feedback from our client on how to improve our layout.
	& Spending enough time going over how to accomplish everything our client wanted and making we followed his guidelines.  \\ \hline

	7
	& We were able to figure out the majority of user stories for the frontend as well as the backend.
	& Not enough time to dedicate to working on the application.
	& Being clear in the future about meet ups and staying on track with everything we said we'd be getting done. \\ \hline

	8
	& Completely figured out our API calls, which will lead to easier to understand objectives.
	& Same as previous week, haven't spent enough time to frontend work.
	& Get to work earlier on the important parts of the project. \\ \hline

	9
	& Steven and I implemented the router, which helped ease the deltas of the previous weeks.
	& No real problems or issues that needed work this week.
	& N/A \\ \hline

	10
	& Clear goals of what we can be working on over the break.
	& No real problems or issues that needed work this week.
	& N/A \\ \hline

\end{tabular}
\clearpage
\section{Conclusion}
With everything that my team has accomplished and worked through this previous and current term, I'm excited to see what we can get done in the next few 
weeks to come. My team has shown that we can get a lot done in a short amount of time, and that puts some of the worries that I could have moving forward 
to rest. I've had a great time working with my group mates, we all seem to have a similiar humor and that makes our meetings fairly interesting. I think 
we'll have a lot to present at expo, and I can't wait to see our finished product. \\

\end{document}
