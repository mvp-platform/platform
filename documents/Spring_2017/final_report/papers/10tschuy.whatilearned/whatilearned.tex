\documentclass[onecolumn, draftclsnofoot,10pt, compsoc]{IEEEtran}

\usepackage{supertabular}

\usepackage{fancyhdr}
\usepackage{csquotes}

\pagestyle{fancy}
\fancyfoot[C]{TSCHUY - WHAT I LEARNED --- \thepage}
\setlength\parindent{0pt}

\begin{document}
\section{What I Learned: Evan Tschuy}

At times fun, at times incredibly frustrating, the last year of capstone has been
a great opportunity for me to get a look at the perils of project management, and
dig into creating a technical product from nothing.

\subsection{What technical information did I learn?}

On a technical level, the project can be broken down into several sections:

\begin{itemize}
	\item a NodeJS backend:
	\begin{itemize}
		\item direct interactions with ElasticSearch
		\item a custom-written library for interacting with \verb|git| repositories
		\item 3rd-party authentication integration with Google
	\end{itemize}
	\item a frontend:
	\begin{itemize}
		\item a single-page application using Aurelia
		\item Materialize for user interface items
	\end{itemize}
\end{itemize}

Every single one of these sections was its own learning experience for me. In the
past, I have written web applications for work. These generally tended to be fairly
self-contained, with simple user interfaces that would fit in well with the year 2009. \\

Instead, this project was hyper-modern. We purposefully chose a single-page application
backed by a modern backend, a complicated datastore, and a powerful, real-world
search engine. All of these things were new to me, and being exposed to them has
prepared me in the event I decide to continue creating web applications.

\subsection{What non-technical information did I learn?}

Some people will need pushing to get things done. There's no way around that, and
so it is simply necessary to know when to push and how. It is necessary all throughout
the year to make tradeoffs in who does what, when, and why.

\subsection{What have I learned about project work?}

It is easier if you just accept that some people put in less effort. After weighing
how much I wanted to finish the project and how much recognition I would get from
my group, my peers, the school, and others, I simply went ahead and did parts that
we had earlier divvied up otherwise. The other option would simply have been to
let the project go uncompleted, which I was not alright with.

\subsection{What have I learned about project management?}

I think that what I learned about project management can be summed up in one small line:

\begin{displayquote}
Divide up work early, and check progress often. Readjust as needed.
\end{displayquote}

\subsection{What have I learned about working in teams?}

Working in teams is a give and take situation. For parts of a project where you
are the most technically familiar, it is often a good idea to take on that section
while someone else works on things they are more familiar with. On the other hand,
if you have a little more free time, it can be good to instead work together on
both sections, so that you can learn what they know and they can learn what you
know.

\end{document}
