\documentclass[onecolumn, draftclsnofoot,10pt, compsoc]{IEEEtran}

\usepackage{supertabular}
\usepackage{hyperref}
\usepackage{fancyhdr}

\pagestyle{fancy}
\fancyfoot[C]{PRODDOC--\thepage}
\begin{document}
\section{Project Documentation}

\subsection{How does it work?}


\subsubsection{Project Structure}

The overall structure of our project relies heavily on JavaScript. Our backend is built on a NodeJS, which handles data to and from a MongoDB database. Our frontend is comprised of Aurelia, which is a JavaScript framework used for single page applications. All three of these components make up the structure of our application, and are constantly being used by each other.

\subsubsection{Theory of Operation}

The MVP Platform is an online application, so navigating to the website is the first step. A user can then sign in using their Google Account, at which point they'll be able to use our application. The whole of our application works in a simple manner, as all a user has to do to get started is click on the "new book" button on the main page. Users can also fork a book on the website if they wish to work off somebody elses work. The next step is to start creating chapters, and then individual scraps from those. A user can add images, paragraphs, or other work to a scrap. The option to take others' scraps and chapters is available as well. When the book is finished, having it reviewed and printed are the only parts left to accomplish.

\subsection{Hardware \& Software Requirements}

\subsubsection{Hardware Prerequisites}

\noindent The MVP Platform is a web application with search and user functionalities requiring a database. As such, it requires a decently powerful machine.

\noindent The platform has been successfully used on a machine with 

\subsubsection{Software Prerequisites}

To begin setup of the MVP Platform, there are a few things that need to be installed.

\begin{itemize}
	\item \href{https://docs.npmjs.com/getting-started/installing-node}{NodeJS and NPM}: NodeJS is the runtime for the application itself, and NPM is the package manager included with it for installing 3rd-party libraries.
	\item \href{https://docs.mongodb.com/manual/installation/}{MongoDB}: This is the database used to store things like user favorites and login tokens.
	\item \href{https://www.elastic.co/guide/en/elasticsearch/guide/current/running-elasticsearch.html}{ElasticSearch}: this is the search engine behind the search feature.
\end{itemize}


\subsection{Installation Instructions}

\subsubsection{Setup}

\begin{enumerate}
	\item Clone the source code: \\
	\hspace*{1.5em}\verb|$ git clone https://github.com/CS461-MVP/mvp_platform.git |
	\item Move to the source code directory: \\
	\hspace*{1.5em}\verb|$ cd mvp_platform/dev|
	\item Install all scrap library dependencies: \\
	\hspace*{1.5em}\verb|$ cd scrapjs && npm install && cd ..|
	\item Install all backend Node dependencies: \\
	\hspace*{1.5em}\verb|$ cd backend && npm install |
\end{enumerate}

\subsubsection{Running the backend}

To run the server, simply run:

\verb|$ npm start| \\

\noindent This will start the server running on localhost, port 8000. To verify that the server is running, in another terminal window run:

\verb|$ curl http://localhost:8000| \\

\noindent This will return a 404, as there is no root page, if the server is successfully running.

\subsubsection{Running the frontend}

The frontend is a static, single-page webapp. To serve this, a standard \verb|nginx| or \verb|Apache| static file server is recommended; alternatively, for testing, the pages can be served with Python's built-in HTTP server:

\begin{verbatim}
$ cd /path/to/clone/mvp_platform/dev/frontend
$ python -m SimpleHTTPServer 8080
\end{verbatim}

\noindent The website may now be visited at \verb|http://localhost:8080|.

\subsubsection{Deploying for Production}

For detailed instructions regarding setting up \verb|Nginx|, creating a service, and setting up a reverse proxy, please find the nearest sysadmin. Such a setup is outside of the scope of this documentation. In brief:

\begin{enumerate}
	\item Create a service that runs the backend as shown above. 
	\item Install \verb|nginx|. 
	\item Create a configuration file that:
	\begin{itemize}
		\item Create a rule that matches requests sent to \verb|/accounts, /login, /favorites, /books,| \\ \verb|/chapters, /scraps, /images, /search, /users|. In this rule, \href{https://www.nginx.com/resources/admin-guide/reverse-proxy/}{reverse proxy} the request to the instance of the platform running on port 8000.
		\item For all other requests, serve static files from \verb|/dev/frontend|.
	\end{itemize}
	\item Restart \verb|nginx| to put the rule into effect.
	
\end{enumerate}


\end{document}
