\documentclass[onecolumn, draftclsnofoot,10pt, compsoc]{IEEEtran}
\usepackage{graphicx}
\usepackage{setspace}
\usepackage{comment}
\usepackage{bigstrut}
\usepackage{geometry}
\usepackage{supertabular}
\usepackage{tabu}
\usepackage{hyperref}
\usepackage{url}
\hypersetup{
  colorlinks=true, linkcolor=blue, citecolor=blue, filecolor=blue, urlcolor=blue}
\geometry{textheight=9.5in, textwidth=7in}

\usepackage{color}
\definecolor{editorGray}{rgb}{0.95, 0.95, 0.95}
\definecolor{editorOcher}{rgb}{1, 0.5, 0} % #FF7F00 -> rgb(239, 169, 0)
\definecolor{editorGreen}{rgb}{0, 0.67, 0} % #007C00 -> rgb(0, 124, 0)

\usepackage{listings}
\lstdefinelanguage{JavaScript}{
  morekeywords={typeof, new, true, false, catch, function, return, null, catch, switch, 
  var, if, in, while, do, else, case, break, addChapter, await},
  morecomment=[s]{/*}{*/},
  morecomment=[l]//,
  morestring=[b]",
  morestring=[b]'
}

\lstset{%
    % Basic design
    backgroundcolor=\color{editorGray},
    basicstyle={\small\ttfamily},   
    frame=l,
    % Line numbers
    xleftmargin={0.75cm},
    numbers=left,
    stepnumber=5,
    firstnumber=1,
    numberfirstline=true,
    % Code design   
    keywordstyle=\color{blue}\bfseries,
    commentstyle=\color{red}\ttfamily,
    ndkeywordstyle=\color{editorOcher}\bfseries,
    stringstyle=\color{editorGreen},
    % Code
    language=JavaScript,
    alsodigit={.:;},
    tabsize=2,
    showtabs=false,
    showspaces=false,
    showstringspaces=true,
    extendedchars=true,
    breaklines=true
}


% 1. Fill in these details
\def \CapstoneTeamName{		Remix}
\def \CapstoneTeamNumber{		61}
\def \GroupMemberOne{			Josh Matteson}
\def \GroupMemberTwo{			Steven Powers}
\def \GroupMemberThree{			Evan Tschuy}
\def \CapstoneProjectName{		Many Voices Platform}
\def \CapstoneSponsorCompany{	Oregon State University}
\def \CapstoneSponsorPerson{		Carlos Jensen}

% 2. Uncomment the appropriate line below so that the document type works
\def \DocType{
        %Problem Statement
				%Requirements Document
				%Technology Review
				%Design Document
				Progress Report
				}

\newcommand{\NameSigPair}[1]{\par
\makebox[2.75in][r]{#1} \hfil 	\makebox[3.25in]{\makebox[2.25in]{\hrulefill} \hfill		\makebox[.75in]{\hrulefill}}
\par\vspace{-12pt} \textit{\tiny\noindent
\makebox[2.75in]{} \hfil		\makebox[3.25in]{\makebox[2.25in][r]{Signature} \hfill	\makebox[.75in][r]{Date}}}}
% 3. If the document is not to be signed, uncomment the RENEWcommand below
\renewcommand{\NameSigPair}[1]{#1}

%%%%%%%%%%%%%%%%%%%%%%%%%%%%%%%%%%%%%%%
\begin{document}
\begin{titlepage}
    \pagenumbering{gobble}
    \begin{singlespace}
    	\includegraphics[height=4cm]{coe_v_spot1}
        \hfill
        % 4. If you have a logo, use this includegraphics command to put it on the coversheet.
        %\includegraphics[height=4cm]{CompanyLogo}
        \par\vspace{.2in}
        \centering
        \scshape{
            \huge CS Capstone \DocType \par
            {\large\today}\par
            \vspace{.5in}
            \textbf{\Huge\CapstoneProjectName}\par
            \vfill
            {\large Prepared for}\par
            \Huge \CapstoneSponsorCompany\par
            \vspace{5pt}
            {\Large\NameSigPair{\CapstoneSponsorPerson}\par}
            {\large Prepared by }\par
            Group\CapstoneTeamNumber\par
            % 5. comment out the line below this one if you do not wish to name your team
            \CapstoneTeamName\par
            \vspace{5pt}
            {\Large
                \NameSigPair{\GroupMemberOne}\par
                \NameSigPair{\GroupMemberTwo}\par
                \NameSigPair{\GroupMemberThree}\par
            }
            \vspace{20pt}
        }
        \begin{abstract}
        % 6. Fill in your abstract
		\noindent This document summaries the progress that the Remix team
		has made on the Many Voices Publishing Platform for the client
		Dr. Carlos Jensen. Additionally this document provides a week by
		week summary of work performed, as overviews of progress that has been made
		throughout Spring Term.
        \end{abstract}
    \end{singlespace}
\end{titlepage}
\newpage
\pagenumbering{arabic}
\tableofcontents
% 7. uncomment this (if applicable). Consider adding a page break.
%\listoffigures
%\listoftables
\clearpage


% 8. now you write!
\section{Revision Log}

\tablehead{}
\begin{supertabular}{|p{3cm}|p{3cm}|p{3cm}|p{7cm}|}
\hline
Name & Change Number & Date & Description of Change
\\\hline
Steven Powers & 1 & 2/17/2017 & Finalized Progress Report for half way point through 
		winter term. Fixed spelling mistakes, added code listings, adjusted content, added images.
\\\hline
\end{supertabular}


\section{Project Purpose}
\noindent A modern textbook is updated frequently, perhaps even yearly,
and can cost in the range of hundreds of dollars. Students are often left
to attempt to understand poorly worded, even incorrect information from a textbook
often chosen from those sent to a professor for review by the publisher.
This can lead to better works with less aggressive sales tactics not being
made available, or even known.
Another choice would be for a professor to write their own textbook.
However, this is a process that takes months of endless research and
time spent, and on top of that will require peer review and publishing
before it can be released. \\

\noindent The Many Voices platform offers to put an end to this massive,
slow, expensive cycle.  Instead of a textbook being a single document
written by one professor, we seek to re-imagine the textbook as instead
a collection of content written by professors from around the world
that are useful for a particular class. A knowledgeable professor can
contribute a few chapters on their specialty, without needing to write
an entire textbook around it. \\

\noindent Professors wishing to use this content can then modify it for
their uses in the classroom. The material will be hosted in such a way
as to provide the ability to "fork" content, or create content based
off of it. The platform will provide a way to search for and find content,
prioritized by relevance and credibility as determined by other users;
the most popular material will be shown with the most prominence.

\newpage

\section{Project Status}

\noindent The beginning of spring term unfortunatey didn't aid us in starting out strong, 
as one of the team members had their laptop stolen over the break. This proved as a 
difficult obstacle to overcome, as we anticipated spring break to be a productive time 
for the application. Thankfully, after two weeks, we were able to recover the stolen 
laptop and start working fluently from then on. While this did manage to put us behind
in a lot of areas, we were able to make a clear plan that we could follow through until 
we had everything checked off. \\

\noindent As for where our project stands now, we managed to meet every one of our requirements. 
The last week before the code freeze proved to be a challenging and time
consuming period, thankfully the majority of the team was able to dedicate many hours to focusing 
on finishing the application. While we weren't able to completely utilize everybody on the 
team due to other classes and obligations, the project satisfactory to our clients expectations.
As of the code freeze, our project is in a polished beta state. We have a fully running version 
of our application, but there is still much more needed testing to do. Users have access to most features 
desired of the full application. \\

\noindent Some of the main functionality for our application is being able to manipulate books, 
chapters, and scraps. This meant a variety of things had to be in place, such as being able to 
remove books, chapters, and scraps. The purpose of our application is to make it easy to write 
and edit books, this means being able to edit individual scraps at the base level. Users are able 
to upload scraps associated with a chapter, and chapters associated with a book. Users are also 
able to write scraps that aren't associated with any chapter but can be used later. What all this
meant for us, is that user would be able to write text, and have it stored in our database as 
LaTeX. Thanks to one of our team members, the backend is fully able to handle images, text, and 
special commands (such as bolding). \\

\noindent One of the big requirements for our application was that it would manage user accounts. 
Users need to have the ability to come back to their unfinished work, and this isn't possible without
user accounts. Our application uses a 3rd party, single sign on tool by Google to manage. Users are 
able to sign in with their Google accounts, and manage certain settings about their profile, like 
their username. When a user creates a book, chapter, or scrap, their name will be associated with it, 
and they'll be able to search for it under their content. Another factor of authentication that needed 
to be implemented was that whenever a user makes an API call, they have a have a varified token ID. \\

\noindent Keeping track of a book, chapter, or scrap in terms of history was a requirement specifically 
asked for by our client. While we weren't able to make the history of an item look similar to the Github
timeline, we were able to incorporate dates and timestamps of events or changes. A user is able to look 
through every action done to a book, chapter and scrap. \\

\noindent There are other miscellaneous tasks the team focused on to improve general quality of life. 
Some of these take place as the buttons below certain items that allow users to pin, favorite, and preview.
"Pinning" an item allows the user to stick to a certain version of that item, making it so that revisions
aren't applied. This is helpful if you want to lock down certain items the way they are without unwanted 
changes being applied to them. Favoriting an item lets the user keep track of said item for their own purpose.
This might look like a chapter being favorited in order to remind the user that they're working on it. 
Previewing allows the user to just focus on a certain item, whether that be an entire book or an individual 
scrap. \\

\noindent

\section{Remaining Work} 

While we were able to accomplish our agreed upon requirements for our client, we weren't able to finish
many of our stretch goals. This was due to a few complications, which will be discussed later in this
report. The majority of the goals discussed are focused on the users quality of life, but will also 
cover whats indicative to our team. This segment will expand upon user feedback, our user manual, and 
miscellaneous features for the users. \\

\noindent Getting thorough and helpful feedback from testers has been a more difficult aspect of the project, as 
initiating and getting responses from people willing to test has been limited. With the amount of time 
we've been given, we've been able to incorporate constructive feedback from family, friends, as well 
as from our client from a professors point of view. This is one area that still needs work and will be
a focus point in the future. \\

\noindent One of the areas our team will focus on in the future in the user manual. The user manual instructs 
the user on how to user all the features offered by our application. This covers everything from making 
a book, to changing your display name. We want our user manual to be a helpful guide for new users to get
acquinted with the site as fast as possible. We'll do this by making the instructions very clear and simple. \\

\noindent Some smaller areas we want to implement in the future have to do with the overall look and appeal
to the application. These include cosmetic and usability features, as well as ways to help give the user
a better experience overall. An example of a cosmetic feature that needs work on is the ability for textareas 
to resize automatically. Being able to view PDF's at the level the user wishes is one of the main areas that
sets our application apart. While we have this feature working at a promising level, we still have several 
details we want to incorporate. Viewing a PDF in fullscreen, and being able to navigate through the pages 
easier would areas the PDF viewer could improve. \\

\noindent The last area that needs to be improved upon is the history feature. As mentioned previously in the
report, we have a history view model that allows the user to see all changes done to it. For the history model 
to be complete, we would like to add more information. This would mean including the commiter or the user who 
performed the specific change. This would help the user get a better understand of who's changing certain 
features about an item. \\

\newpage

\section{Problems and Proposed Solutions} 

\noindent One of the areas that our team wanted to improve on winter term was communication with our client.
This meant that we needed to come to our meetings with an agenda and clear goals of what we wanted to 
accomplish. After much discussion as to what was expected, both us and our client were able to arrive at 
reasonable goals. While our communication has improved with our client, we still have difficulties finding 
meetings times that work for our client and us. The solution that we have incorporated is being proactive
about our meetings with our client anddoing everything we can to not make it difficult or a hassle.\\

\noindent A problem our group has ran into in the past, as well as currently, is members not always being 
able to contribute to the project as much as necessary. This has impeded progress in several cases, and 
has caused the other team members to feel unvalued and unappreciated. Our proposed solution to this problem
has been having the team members who weren't able to contribute as much take up responsibility for the next
big project, or give them more task in order to make up for their lack thereof. \\

\newpage

\section{Steven Term Progress}

\subsection{Weekly Recapping}

\section{Evan Term Progress}

\subsection{Weekly Recapping}

\section{Josh Term Progress}
\subsection{Weekly Recapping}

\section{Conclusion}

\end{document}