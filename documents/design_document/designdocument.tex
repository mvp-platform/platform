\documentclass[letterpaper, 10pt, draftclsnofoot, compsoc, onecolumn]{IEEEtran}

\usepackage{graphicx}
%\usepackage{amssymb}
%\usepackage{amsmath}
%\usepackage{amsthm}

\usepackage{alltt}
\usepackage{float}
\usepackage{url}

\usepackage{balance}
\usepackage[TABBOTCAP, tight]{subfigure}
\usepackage{enumitem}
\usepackage{pstricks, pst-node}

\usepackage{hyperref}
\usepackage{geometry}

\usepackage{comment}
\setcounter{tocdepth}{4}
\setcounter{secnumdepth}{4}

\usepackage{listings}
\usepackage{color}
\usepackage[doublespacing]{setspace}
\usepackage{supertabular}
\usepackage{fancyhdr}

%\setlength{\parindent}{4em}
%\linespread{1.1}

\pagestyle{empty}
\renewcommand{\headrulewidth}{0pt}
\lhead{Many Voices Publishing Platform}

\lfoot{DD}
\cfoot{Page}
\rfoot{\thepage}

\def\name{Steven Powers, Josh Matteson, Evan Tschuy}

%pull in the necessary preamble matter for pygments output
%\input{pygments.tex}

% The following metadata will show up in the PDF properties
\hypersetup{
  colorlinks=true, linkcolor=blue,citecolor=blue, filecolor=blue,
  urlcolor=blue, pdfauthor = {\name},
  pdfkeywords = {cs461 ``senior capstone'' Design Document},
  pdftitle = {CS 461 Design Document},
  pdfsubject = {CS 461 Design Document},
  pdfpagemode = UseNone
}





\begin{document}


\begin{titlepage}
\centering
{\huge Many Voices Publishing Platform\par}
{\LARGE Design Document\par}
{\vspace{5mm}}
{\large D. Kevin McGrath \& Dr. Kirsten Winters -  CS461 Fall 2016\par}
{\large Commix\par}
{\large Steven Powers, Josh Matteson, Evan Tschuy\par}
{\vspace{10mm}}

\begin{abstract}
\noindent The Many Voices Publishing Platform uses a variety of technologies to handle different aspects of the project, from the user interface to the backend database operations. These technologies enable to the Many Voices Publishing Platform to succeed in delivering a working platform for textbook collaboration.
\end{abstract}

\end{titlepage}

\tableofcontents

\newpage

\setcounter{page}{1}\pagestyle{fancy}



%Section 1
\vspace{1pc}
\section{Introduction}


%Section 2
\vspace{1pc}
\section{Overview}

\vspace{1pc}
\subsection{Scope}
\vspace{1pc}

{\noindent \par}

\vspace{1pc}
\subsection{Purpose}
\vspace{1pc}
{\noindent \par}

\vspace{1pc}
\subsection{Intended audience}
\vspace{1pc}
{\noindent \par}

\vspace{1pc}
\subsection{Conformance}
\vspace{1pc}
{\noindent \par}




%Section 3
\vspace{1pc}
\section{Definitions}




%Section 4
\vspace{1pc}
\section{Conceptual model for software design descriptions}
\vspace{1pc}
{\noindent \par}


\vspace{1pc}
\subsection{Software design in context}
\vspace{1pc}
{\noindent \par}

\vspace{1pc}
\subsection{Software design descriptions within the life cycle}
\vspace{1pc}
{\noindent \par}



%Section 5
\vspace{1pc}
\section{Design description information content}
\vspace{1pc}
{\noindent \par}

\vspace{1pc}
\subsection{Introduction}
\vspace{1pc}
{\noindent \par}

\vspace{1pc}
\subsection{SDD identification}
\vspace{1pc}
{\noindent \par}

\vspace{1pc}
\subsection{Design stakeholders and their concerns}
\vspace{1pc}
{\noindent \par}

\vspace{1pc}
\subsection{Design views}
\vspace{1pc}
{\noindent \par}

\vspace{1pc}
\subsection{Design viewpoints}
\vspace{1pc}
{\noindent \par}

\vspace{1pc}
\subsection{Design elements}
\vspace{1pc}
{\noindent \par}

\vspace{1pc}
\subsection{Design overlays}
\vspace{1pc}
{\noindent \par}

\vspace{1pc}
\subsection{Design rationale}
\vspace{1pc}
{\noindent \par}

\vspace{1pc}
\subsection{Design languages}
\vspace{1pc}
{\noindent \par}






%Section 6
\vspace{1pc}
\section{Design viewpoints}
\vspace{1pc}
{\noindent \par}

\vspace{1pc}
\subsection{Introduction}
\vspace{1pc}
{\noindent \par}

\vspace{1pc}
\subsection{Context viewpoint}
\vspace{1pc}
{\noindent \par}

\vspace{1pc}
\subsection{Composition viewpoint}
\vspace{1pc}
{\noindent \par}

\vspace{1pc}
\subsection{Logical viewpoint}
\vspace{1pc}
{\noindent \par}

\vspace{1pc}
\subsection{Dependency viewpoint}
\vspace{1pc}
{\noindent \par}

\vspace{1pc}
\subsection{Information viewpoint}
\vspace{1pc}
{\noindent \par}

\vspace{1pc}
\subsection{Patterns use viewpoints}
\vspace{1pc}
{\noindent \par}

\vspace{1pc}
\subsection{Interface viewpoint}
\vspace{1pc}
{\noindent \par}

\vspace{1pc}
\subsection{State dynamics viewpoint}
\vspace{1pc}
{\noindent \par}

\vspace{1pc}
\subsection{Algorithm viewpoint}
\vspace{1pc}
{\noindent \par}

\vspace{1pc}
\subsection{Resource viewpoint}
\vspace{1pc}
{\noindent \par}



%Section 7
\vspace{1pc}
\section{Annex A - (information Bibliography}
\vspace{1pc}

\bibliographystyle{IEEEtran}
%\bibliography{designdocument}


%Section 8
\vspace{1pc}
\section{Annex B - Conforming design language description}
\vspace{1pc}
{\noindent \par}

%Section 9
\vspace{1pc}
\section{Annex C - Templates for an SDD}
\vspace{1pc}
{\noindent \par}


%Section 10
\newpage
\section{Conclusion}
{\noindent  The Many Voices Publishing Platform is a combination of User Interfaces, Documentation, User Centered Design, Testing, User Authentication, Databases, Server Back-end, Text Formatting, Password Storage, and the users themselves. Determining the technologies behind these parts and pieces is a difficult task to accomplish, as many choices can satisfy the requirements of the project. Finding the best solution however is the goal of this document, to provide a clear path forward for the platform as a whole. \par}

%\newpage
%\bibliographystyle{IEEEtran}
%\bibliography{designdocument}


\end{document}
