\documentclass[onecolumn, draftclsnofoot,10pt, compsoc]{IEEEtran}
\usepackage{graphicx}
\usepackage{url}
\usepackage{setspace}
\usepackage{comment}
\usepackage{bigstrut}
\usepackage{geometry}
\usepackage{tabu}

\geometry{textheight=9.5in, textwidth=7in}

% 1. Fill in these details
\def \CapstoneTeamName{		Remix}
\def \CapstoneTeamNumber{		61}
\def \GroupMemberOne{			Josh Matteson}
\def \GroupMemberTwo{			Steven Powers}
\def \GroupMemberThree{			Evan Tschuy}
\def \CapstoneProjectName{		Many Voices Platform}
\def \CapstoneSponsorCompany{	Oregon State University}
\def \CapstoneSponsorPerson{		Carlos Jensen}

% 2. Uncomment the appropriate line below so that the document type works
\def \DocType{
        %Problem Statement
				%Requirements Document
				%Technology Review
				%Design Document
				Progress Report
				}

\newcommand{\NameSigPair}[1]{\par
\makebox[2.75in][r]{#1} \hfil 	\makebox[3.25in]{\makebox[2.25in]{\hrulefill} \hfill		\makebox[.75in]{\hrulefill}}
\par\vspace{-12pt} \textit{\tiny\noindent
\makebox[2.75in]{} \hfil		\makebox[3.25in]{\makebox[2.25in][r]{Signature} \hfill	\makebox[.75in][r]{Date}}}}
% 3. If the document is not to be signed, uncomment the RENEWcommand below
\renewcommand{\NameSigPair}[1]{#1}

%%%%%%%%%%%%%%%%%%%%%%%%%%%%%%%%%%%%%%%
\begin{document}
\begin{titlepage}
    \pagenumbering{gobble}
    \begin{singlespace}
    	\includegraphics[height=4cm]{coe_v_spot1}
        \hfill
        % 4. If you have a logo, use this includegraphics command to put it on the coversheet.
        %\includegraphics[height=4cm]{CompanyLogo}
        \par\vspace{.2in}
        \centering
        \scshape{
            \huge CS Capstone \DocType \par
            {\large\today}\par
            \vspace{.5in}
            \textbf{\Huge\CapstoneProjectName}\par
            \vfill
            {\large Prepared for}\par
            \Huge \CapstoneSponsorCompany\par
            \vspace{5pt}
            {\Large\NameSigPair{\CapstoneSponsorPerson}\par}
            {\large Prepared by }\par
            Group\CapstoneTeamNumber\par
            % 5. comment out the line below this one if you do not wish to name your team
            \CapstoneTeamName\par
            \vspace{5pt}
            {\Large
                \NameSigPair{\GroupMemberOne}\par
                \NameSigPair{\GroupMemberTwo}\par
                \NameSigPair{\GroupMemberThree}\par
            }
            \vspace{20pt}
        }
        \begin{abstract}
        % 6. Fill in your abstract
This document summaries the progress that the Remix team has made on the Many Voices Publishing Platform for the client Dr. Carlos Jensen. Additionally this document provides a week by week summary of work performed, as well as what is needed to be changed to improve effectiveness in building the MVP platform.
        \end{abstract}
    \end{singlespace}
\end{titlepage}
\newpage
\pagenumbering{arabic}
\tableofcontents
% 7. uncomment this (if applicable). Consider adding a page break.
%\listoffigures
%\listoftables
\clearpage


% 8. now you write!
\section{Project Purpose}
\noindent A modern textbook is updated frequently, perhaps even yearly, and can cost in the range of hundreds of dollars. 
Students are often left to attempt to understand poorly worded, even incorrect information from a textbook 
often chosen from those sent to a professor for review by the publisher. This can lead to better works with 
less aggressive sales tactics not being made available, or even known. Another choice would be for a 
professor to write their own textbook. However, this is a process that takes months of endless research 
and time spent, and on top of that will require peer review and publishing before it can be released. \\

\noindent The Many Voices platform offers to put an end to this massive, slow, expensive cycle.  
Instead of a textbook being a single document written by one professor, we seek to re-imagine 
the textbook as instead a collection of content written by professors from around the world 
that are useful for a particular class. A knowledgeable professor can contribute a few chapters 
on their specialty, without needing to write an entire textbook around it. \\

\noindent Professors wishing to use this content can then modify it for their uses in the classroom. 
The material will be hosted in such a way as to provide the ability to \"fork\" content, or 
create content based off of it. The platform will provide a way to search for and find content, 
prioritized by relevance and credibility as determined by other users; the most popular material 
will be shown with the most prominence. 

\section{Weekly Updates}

\subsection{Week 3}
The term started with ironing out the problem statement. At the end of the 
second week, we met with our client, Dr. Jensen, and briefly went over exactly 
what his vision was for the end result of the platform. The third week was 
then spent on the initial drafting of the problem statement, which was turned 
in at the end of the week. Unfortunately, we were unable to meet with Carlos 
again to go over our draft, as he was unavailable.

\subsection{Week 4}
We attempted to get Carlos' feedback on our problem statement document, 
and get him involved in the creation of the coming requirements document, 
but were unfortunately again unable to find a meeting time that worked 
well for him. Therefore, we simply pressed on and began the revision process 
using the feedback our TA, Jon Dodge. We spent some of the time investigating 
tools that would possibly be useful in the development phase of the product.

\subsection{Week 5}
This week, we managed to get a hold of our client and get him to sign our 
revised Problem Statement. We previously were having quite a bit of trouble 
getting a hold of him because he's been traveling. We also managed to all 
work together on our Requirements Document, which has been one of the few 
times we've all been able to find times in our schedule to do it together 
(even though it was over the internet). The Requirements document was 
pretty intimidating looking, simply by virtue of its size!

\subsection{Week 6}
During week six, Steven made a lot of progress revising the requirements document, 
and then integrated changes. Evan and Josh made to the document. He was also able 
to attend Dr. Winters' writing session, where we received useful and actionable 
feedback about documentation formatting. We also revised the Problem Statement 
after hearing from Dr. Winters that we may be able to re-submit revised documents 
for regrading. This will hopefully prevent us from receiving another 82/100. 
Two drafts were sent out to our client, with 48 hours notice each time, but 
unfortunately was only able to get a signature on the second a few hours before 
turn-in time. The main issue we had this week were slight differences in the 
requirements document formatting compared to the IEEE 830 format specification, 
though our TA Jon Dodge and Dr. Winters feel that the document looks great. 
It was also slightly concerning that we were not able to get feedback apart 
from the signature of our client, though this is understandable because he is 
traveling right now.


\subsection{Week 7 \& 8}
The technology review document was the main point of focus during Week 7, with 
it being due at end-of-day Monday of Week 8. Steven also made revisions to the 
requirements document, including many suggested by Jon Dodge, to prepare for a 
re-submission for regrading. Evan was not available for a good portion of the 
week seven, as he was on an important trip to California, but as we split the 
document into its constituent sections early in the week, he was able to work 
on his way down. On Monday, we combined the efforts that had been written up 
to that point. Unfortunately, we were not satisfied with the state of the 
document, and so we requested an extension. We were, along with the rest of 
the class, given an additional 36 hours, pushing the due date back to Wednesday 
at noon, when we turned in a satisfactory document.


\subsection{Week 9 \& 10}
Week 9 was mainly spent with our respective families, and as a short break from 
the march towards project completion. We made the strategic decision to focus 
on getting rested and well-fed in an attempt to mentally prepare ourselves for 
the last two main parts of the Fall-term requirements. \\
 
\noindent Week 10 began with work towards finishing our design document. After a few false 
starts and a night of less sleep than desired, the final design document was turned 
in on time. Unfortunately, we were unable to get our client's signature in time; 
we will be providing a signed copy to be graded as soon as possible. Later in Week 10, 
during the weekend before finals week, we wrote our progress update document and 
recorded our progress update video.


\section{Weekly Summary}

\begin{tabular}{|p{0.05\linewidth}|p{0.285\linewidth}|p{0.285\linewidth}|p{0.285\linewidth}|}
\hline 
Week & Positives & Deltas & Actions \\ \hline

3 &
Got the initial draft of the problem statement done after meeting last 
	week with our client. &
Were unable to set up a meeting with our client, and thus had no way of 
	getting feedback on the draft before submission. &
We will talk to our client via email about possible meeting times, either 
	in person or remote, that may possibly work for everyone. \\ \hline

4 &
Figured out a time for meeting with our TA. &
Need to hear back from our client. &
We will wait another day or two and then send a new polite email. \\ \hline

5 &
Finished revising the problem statement, got it signed. Began the requirements document. & N/A & N/A\\ \hline

6 &
Revised Requirements document via feedback from TA. Got client signature on Requirements. 
	Attended writing workshop. &
Need to be on top of things regarding contacting our client as he is difficult to contact. &
We will send him another email asking for feedback on the requirements. \\ \hline

7 &
Began technology review. Revised requirements document for possible re-grading. &
Need better team work schedule to get times to work on documents together rather than separately. &
We will discuss possible extra meeting times in addition to our weekly meeting 
	with our TA where we can work for several hours uninterrupted. \\ \hline

8 &
Finished technology review (after getting extension to Wednesday). &
Need to be on top of documents to finish them well in advance of due dates! &
We will personally push ourselves to get work done sooner rather than later. \\ \hline


9 \& 10 &
Spent Thanksgiving Week getting rested. Finished design document. 
	Finished progress update document/finished presentation. &
Need to get client signature for design document! &
We will be emailing Carlos. With the new term we hope to set up a weekly meeting with him.  \\ \hline

\hline
\end{tabular}

\section{Retrospective}

\begin{tabular}{|p{0.3\linewidth}||p{0.3\linewidth}|p{0.3\linewidth}|}
\hline
Positives & Deltas & Actions \\ \hline

Team came together on planning and design &
Client communication &
We will talk to our client via email about possible meeting times, either in person or remote, that may possibly work for everyone. \\ \hline

Learned a lot about the software development process &
Documentation / Development Confusion &
If we are confused and blocked by something in the class that could be helped by asking a question of either our TA Jon Dodge or Professors D. Kevin McGrath or Dr. Kirsten M. Winters, we will. \\[0.000001cm] \hline

Learned a lot about Latex and the writing of technical documents. &
Team Communication &
Solved: Problems with communication were solved by transition to Slack \& Email communication \\[1pt]  \hline

&
Team Meeting Time &
Need to compare our schedules and find a time that will work for all of us to get together. \\ \hline

\hline
\end{tabular}

\section{Current Project Status}

Thus far we have written documentation detailing the technical and design requirements of the project.
Now, we are beginning to move into the technical development phase.
To begin, we are finishing coalescing a unified vision of what the project is, and how we will go about architecting and building it.
We are planning on beginning to build our initial prototype over winter break, based on the basic skeleton laid out in the documentation. \\

\noindent Moving forward, we expect to spend the majority of time doing individual development work, with weekly team development sessions to keep ourselves on the same track.
In doing so we will be able to progress even when one individual team member is blocked on either something relating to the project, or on other work.
To do this, we have split the project into chunks, which we will then work on either solo or in a pair. 

\section{Impending Problems}

For Fall term, client communication was a problem that impeded our progress at a few points.
Due to busy schedules of the team and the client, communication slipped from where we expected it to be.
This caused problems with project requirements and other questions the team had about moving forward.
For Winter term and beyond, the team plans on having a weekly meeting with our client and additionally providing a weekly email detailing our progress of the week. \\ 

\noindent Additionally, finding time for the team to come together to work on the project proved difficult.
Our schedules had many conflicting times with classes and work times that made it difficult to spend large portions of time together.
This resulted in a lot of remote development of the planning and documentation, resulting in less detailed documentation and lower scores on grading.
In the recent weeks, the team has spent more time working together, which has led to more cohesive development.
For Winter term and beyond, the team has schedules that align more cohesively, allowing for more time to be spent discussing and developing the platform together.

\section{Conclusion}

The Many Voices Publishing Platform has been a great project to work on, bringing each team member outside their comfort zone. A lot of planning has taken place over Fall term, sometimes resulting in shifts in direction of how to manage and develop the platform. The Remix team feels more comfortable moving into Winter term, and for development to begin on the platform. Part of this comfort comes from the planning of weekly meetings with the client, and weekly emails to detail the current project status.

\end{document}
